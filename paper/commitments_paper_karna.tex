%
\documentclass[11pt,english]{article}
\usepackage[T1]{fontenc}
\usepackage[latin9]{inputenc}
\usepackage{hyperref}
\usepackage{geometry}
\geometry{verbose,tmargin=1.25in,bmargin=1.25in,lmargin=1.25in,rmargin=1.25in}
\usepackage{babel}
\usepackage{url}
\usepackage{amsmath}
\usepackage{amsthm}
\usepackage{amssymb}
\usepackage{graphicx}
\usepackage{setspace}
\usepackage{wasysym}
\usepackage[authoryear, longnamesfirst]{natbib}
\onehalfspacing
\usepackage{breakurl}
\usepackage{csquotes}
\usepackage{enumitem}
\setcitestyle{authoryear, open={((},close={)}}


\makeatletter
%%%%%%%%%%%%%%%%%%%%%%%%%%%%%% Textclass specific LaTeX commands.
  \theoremstyle{plain}
  \newtheorem{prop}{\protect\propositionname}
 \theoremstyle{definition}
 \newtheorem*{defn*}{\protect\definitionname}

%%%%%%%%%%%%%%%%%%%%%%%%%%%%%% User specified LaTeX commands.
\usepackage{babel}
\usepackage{babel}
\date{July 28, 2021}\usepackage{babel}
\usepackage{babel}

\makeatother

  \providecommand{\definitionname}{Definition}
  \providecommand{\propositionname}{Proposition}


\begin{document}

%\shorthandoff{=}

\title{Commitments under Threat:\\Contracting with Present-Biased Consumers\\amidst Renegotiation Concerns}

\author{Karna Basu \& Jonathan Conning\thanks{Department of Economics, Hunter College \& The Graduate Center, City
University of New York. Email: \href{mailto:kbasu@hunter.cuny.edu}{kbasu@hunter.cuny.edu}, \href{mailto:jconning@hunter.cuny.edu}{jconning@hunter.cuny.edu}.
We are grateful to the Roosevelt House Public Policy Institute and
a Small Business Administration research grant for support. For detailed
comments and suggestions, we thank Temisan Agbeyegbe, Partha Deb, Maitreesh Ghatak, Alexander Karaivanov, Igor Livshits, Mihir Sharma, and especially Abhijit Banerjee and Eric Van Tassel; conference participants at the NBER Development
Economics Summer Institute, "Economics of Social Sector Organizations"
(Chicago Booth), CIDE-ThReD Conference on Development Theory, ACEGD
(ISI Delhi) and NEUDC (Harvard Kennedy School); and seminar participants
at Michigan State University, IIES Stockholm, Delhi School of Economics,
Queens College, Indian School of Business, Hunter College, Florida
Atlantic University, Pontificia Universidad Javeriana, and The CUNY Graduate
Center. Kwabena Donkor provided excellent early research assistance.
Replication files for figures and other supplemental materials at:
\protect\url{https://jhconning.github.io/commitments/}.}}




\maketitle

\begin{abstract}
Hyperbolic discounters value consumption-smoothing commitment contracts, but may fear that these could be renegotiated by future selves and banks. This creates a consumer protection problem even for sophisticated and informed consumers. This paper studies how the threat of renegotiation affects equilibrium commitment contracts and bank governance. We find that familiar behaviors such as "over"-borrowing or "under"-saving emerge, but here they are a rational response to the threat of renegotiation. We then show how it may be to banks' advantage to offer additional consumer protection either via an appeal for government regulation or through costly private governance/ownership choices. By restricting their own ability to profit from opportunistic renegotiation, banks can expand gains to trade and captured profits. We thus establish new behavioral micro-foundations for a theory of commercial non-profits. The framework helps explain patterns of contracting and ownership forms in consumer banking and microfinance, and how these co-evolve with market structure.   
\\JEL Codes: O16, D03,D18 
\\Keywords: Time-Inconsistency, Commitment, Renegotiation, Nonprofits
\end{abstract}
\vspace{\baselineskip}

\section{Introduction}

Hyperbolic discounters \textendash{} consumers with present-biased and dynamically inconsistent preferences \textendash{} struggle to stick to long-term plans. A substantial experimental and observational literature provides evidence that behaviors consistent with such preferences are widespread.\footnote{See for example \citet{ariely2002}, \citet{thaler2004}, \citet{ashraf2006}, and \citet{bauer2012}. \citet{bryan2010}
survey this empirical literature.} If this is so, a few reasonable and important questions follow: When will financial intermediaries supply the commitment services that may help present-biased consumers stick to long-term savings accumulation or debt management plans? When might they, instead, pander to or otherwise exploit those same consumer biases for opportunistic profit? More broadly, how much financial trade is lost or distorted to such concerns?

\citet{strotz1956} was first to formalize the idea that \emph{sophisticated} hyperbolic consumers \textendash{} those who correctly understand how their own changing preferences would lead future selves to try to undo earlier laid consumption plans \textendash{} might demand and benefit from contracts and other commitment devices to constrain
their future choices.\footnote{For related reasons, naive consumers, who underestimate how their future preferences will change, may also be advantaged by certain public regulations or certain forms of private paternalism of organizations that constrain individuals' actions \citep{spiegler2011}.} Many elements of financial arrangements, ranging from automatic payroll deduction retirement savings or mortgage payment plans, to high-frequency
repayment and joint-liability provisions in microcredit, have  been interpreted as commitment mechanisms designed to deter changes to contracted long-term plans. Experimental
evidence has demonstrated positive take-up and asset accumulation effects following the introduction of new financial commitment products.

The same evidence, however, can be turned around to highlight the apparent puzzle of why, if such contract innovations offer such benefits, they are not more prevalent. In one explanation, \citet{bernheim2015} argue that external commitment devices could weaken \emph{internal} self-control mechanisms. In another, \citet{laibson2015} shows that plausible levels of risk, naivete, and costs of adoption could together significantly dampen the attractiveness of commitment.

The above papers explain diminished demand for commitment contracts. We introduce a consideration that makes contracting for their \textit{supply} also be difficult and costly. In particular, why should a financial intermediary's own promise to help the consumer remain committed to the terms of a contract be credible and not itself be renegotiated? A bank understands that the consumer who demands commitment contracts in one period may, in later periods and with new preferences, willingly pay to refinance or renegotiate its terms \textendash{} with the same bank or a new one. When this is the case, pandering to the consumer's preference changes may increase bank profits, and since the consumer's earlier self is no longer around to protest, most courts would judge such renegotiations as voluntary and legal.\footnote{In most countries, including the United States, courts do not penalize voluntary renegotiation, on the principle that there is no injured promisee \citep[see discussion in][p448]{laibson1997}. We further discuss the issue of welfare judgments below.} Effective commitment contracts must therefore tie the hands not just of consumers but of the intermediaries who serve them.

In this paper we formally examine the role played by the threat of renegotiation in the design of commitment contracts. We show how this threat, and the consumer's anticipation of this threat, limits the scope of commitment, shapes equilibrium contracts, and informs firm governance and ownership decisions. We demonstrate how results depend crucially on consumer type (sophisticated or naive), market structure (monopoly or competition), and costs of renegotiation.

In doing so, we step away from viewing contracts through the lens of "commitment vs no-commitment." Rather, we show how parties must dilute commitment contracts to adapt to exogenous renegotiation threats. The adoption of such contracts then is reflective not of a lower demand for commitment but of the recognition that more stringent commitment is contractually infeasible.

A rich literature has addressed contract design (with time-consistent agents) when renegotiation is possible.\footnote{See \citet{hart1988}, \citet{dewatripont1989}, \citet{bolton1990}, and \citet{rubinstein1992}.} The issue is  summarized succinctly by \citet[p303]{bolton1990}:
\begin{quote}
Basically, the possibility of renegotiation amounts to the addition of another constraint on the set of feasible contracts: now contracts must be not only incentive compatible but also renegotiation-proof. (When parties can commit not to renegotiate they have a choice of when to allow for renegotiation and when not. If this commitment possibility is withdrawn they are forced to renegotiate whenever there are ex-post gains from renegotiation. Since the outcome of this renegotiation is perfectly predictable they might as well write renegotiation-proof
contracts).
\end{quote}

Similar renegotiation concerns are also inherent to commitment contracts between intermediaries (such as banks) and hyperbolic discounters. Here, the concerns arise from consumers' time-inconsistent preferences and not from asymmetric information as in the literature cited above. 

The distinction between a commitment technology and its accompanying renegotiation concern is a subtle one. To illustrate, we consider two classic studies --  \citet{laibson1997} and \citet{ashraf2006}. In the former, the consumer may invest in an illiquid asset that can only be liquidated with a lag. In the latter, the consumer is offered a savings account where deposits remain illiquid until a specified balance or date is reached.

These describe the commitment technology -- the extent to which a consumer's hands can be tied in a contract. But in themselves they do not speak to the possibilities for renegotiation -- the extent to which a contract can be unraveled by mutual agreement. To extend the \citet{ashraf2006} analogy of Ulysses being tied to the mast as a form of commitment against the pull of the sirens: the effectiveness of commitment depends not just on the strength of the rope that Ulysses' crew uses to tie him, but also on the extent to which he can renegotiate with the same crew to untie him. If renegotiation is easy, commitment becomes infeasible regardless of the strength of the technology.

This is a non-trivial concern. After all, by definition, a time-inconsistent consumer who values commitment must later seek ways to undo it, by either renegotiating with the same intermediary (e.g. bribing the bank officials in \citealp{ashraf2006}) or contracting with a new intermediary (e.g. taking short-term loans against the illiquid asset in \citealp{laibson1997}).


The above papers quite reasonably set aside the question of renegotiation; indeed, one could argue that the threat of renegotiation has been implicitly dealt with in their descriptions of the commitment technology. In the current paper, we make this distinction explicit. We parametrize the threat of renegotiation through an exogenous renegotiation cost, and identify new strategic considerations that deliver some theoretical and practical insights. Under the assumption of CRRA utility we derive closed form contract solutions and teachable graphical demonstrations of mechanisms and tradeoffs. The tractable framework allows us to pinpoint mechanisms and contract design features that have sometimes been obscured or understudied in previous work.

We highlight two results before proceeding. First, consider behaviors such as seemingly excessive borrowing or inadequate saving, which are typically (and often with good reason) interpreted as some form of consumer exploitation by predatory firms. We show that similar outcomes can also emerge as a rational endogenous response to the threat of renegotiation. 

Second, we show how firms may pursue costly strategies to provide their own forms of informal consumer protection, for instance by adopting nonprofit or hybrid ownership status (e.g. bringing on social investors), that serve as a commitment to not engage in opportunistic renegotiation. We thus offer an explanation for commercial nonprofits similar to that by \citet{hansmann1996a} and formalized by \citet{glaeser2001}, but set here on different behavioral micro-foundations with no appeal to asymmetric information or naivete.\footnote{Section 7 contains a more detailed review of relevant literature and empirical evidence.}  We study where such endogenous consumer protection mechanisms are most likely to emerge or be undermined. Similar to the modern theory of prudential regulation of banks, consumer protection policy should aim to understand and harness the mechanisms that give banks incentives to regulate their own opportunistic behavior \citep{dewatripont1999}. 

An outline of the paper follows.

A hyperbolic discounter\footnote{The consumer is formally modeled as a \emph{quasi}-hyperbolic discounter, but we generally drop the prefix for expositional convenience.} wishes to rearrange her income stream to achieve consumption smoothing. A bank possesses the technology to commit the consumer to a contracted consumption plan. In Section 3, we characterize the first-best "full-smoothing commitment contracts" that are arrived at in the absence of renegotiation concerns.\footnote{Under hyperbolic discounting, there is no obvious measure of welfare. Our focus is on how renegotiation concerns affect the initial contract. Accordingly, we define "full smoothing" or "first-best" as the contract that maximizes the discounted utility of the initial signatory.}

In Section 4, we establish conditions under which the full-smoothing commitment contract survives the threat of renegotiation. When the consumer is sophisticated, any contract must satisfy a no-renegotiation constraint -- a requirement that limits the feasible contract set to those that sufficiently reduce the bank's and consumer's later selves' gains to future renegotiation as to prevent renegotiation from happening. 

Let the exogenous cost (technological, legal, or psychic) of renegotiation be given by $\kappa$. If this cost is above a threshold level $\bar{\kappa}$ (that is equivalent to the maximal gains that could be transferred to the bank by later renegotiating), then the first-best contract can be sustained and the no-renegotiation constraint is met with slack. 

All else equal, full-smoothing is relatively more feasible under monopoly than under competition. This is due not to the monopolist's superior ability to commit, but to the fact that monopolists leave less surplus with consumers to begin with. As a result the potential future gains from renegotiation are also lower, making commitment easier to maintain.

The more interesting question pertains to how contracts will be adapted as the exogenous penalty $\kappa$ falls below this threshold $\bar{\kappa}$. This is studied in Section 5. The no-renegotiation constraint then binds and the parties will be forced to distort the terms of their contracts to create endogenous incentives to sustain commitment that substitute for the reduced availability of exogenous penalties. The contracted consumption-savings path must be adjusted just enough in the direction of accommodating future selves' preferences that renegotiation does not provide incremental gains large enough to make them want to incur those costs. This second-best "imperfect-smoothing commitment contract" represents the necessary concession to future preferences to make future renegotiation unprofitable.

Under monopoly, the possibility of renegotiation results in \emph{larger} loans for sophisticated consumers but \emph{smaller} loans for naive consumers. Large loans are in practice often viewed as opening the door to exploitation but in this case we isolate a
mechanism where the opposite is true. Intuitively, larger loans today reduce future surplus and make renegotiation less attractive (good for sophisticates), while the opposite is true for smaller loans (allowing the bank to make a second round of profits from renegotiation). 

In Section 6, we allow firms to make strategic firm ownership and capital structure choices as a costly strategy to provide endogenous consumer protection. By operating as a nonprofit (or as a "hybrid" ownership bank), the bank agrees to legal or governance restrictions on how profits generated from any such opportunistic renegotiation may be distributed and enjoyed. Such choices can credibly assure the sophisticated consumer that the bank is less likely to renegotiate the contract in the future, raising contracting surplus and therefore how much can be ultimately extracted by the bank's stakeholders. Nonprofit and hybrid ownership firms can hence survive and compete even in the absence of motivated agents or asymmetric information. Similarly, a firm that credibly commits to sharing its profits with the consumers or community it serves (say, as a form of "corporate social responsibility") could see its net profits rise.

The firm's decision strategy rests on a trade-off, however, which itself is sensitive to market structure and the exclusivity of contracts. Importantly, the proliferation of nonprofit banks could diminish those same banks' incentive to remain nonprofits. 

Finally, Section 7 discusses relevant literature and  applications. How important are these issues? As judged by the often loaded language employed in the popular press and academic writing (particularly legal scholarship) there appears to be a fairly widespread perception that failures of commitment can distort financial behaviors and create socially destructive outcomes. This is suggested by terms such as "overindebtedness" in the market for microcredit, "debt traps" and "excessive debt" rollovers in payday lending, or the "raiding of equity or savings" or "excessive refinancing" of home mortgage loans. Blame for these perceived problems is variously placed on the consumer or the financial intermediary: on consumers for supposedly exhibiting present-bias and weak self-control, and on financial intermediaries for opportunistically exploiting those consumer biases using possibly deceptive methods. Apparent misbehavior by financial intermediaries is, in turn, often attributed to "failures of governance" or the failures of regulation to provide sufficient consumer protections and market policing. In the United States, legal scholar Elizabeth Warren rose to the status of Senator and presidential candidate based in part on her fame crusading for consumer financial protections to limit problems of this sort \citep{sullivan2000}.

Although consumer protection analyses are often framed in terms of the need to protect naive hyperbolic discounters who might fail to understand how their own changing future preferences could leave them vulnerable to exploitation, financial firms themselves often acknowledge they will lose business and profits unless they address consumer protection concerns and put a check on certain types of destructive competition. Microfinance
industry funded initiatives such as the "Smart Campaign",\footnote{https://www.smartcampaign.org} launched with marketing slogans such as "{[}p{]}rotecting clients is not only the right thing to do; it's the smart thing to do," aim to get financial intermediaries to publicly pledge to consumer protection principles and outside audits to prevent aggressive loan sales and client overindebtedness ("overindebtedness" as debt that might accumulate through renegotiation or rollovers of initial contracts).

Our framework extends the simple "for-profit/non-profit" dichotomy of some earlier commercial non-profit literature to explore a fuller spectrum of hybrid ownership firms, for example for-profit intermediaries partly owned and controlled by "social" and double-bottom line investors. This helps makes sense of the ownership and capital structures observed historically in consumer banking in the United States and other now developed countries, as well as microfinance in developing countries where to this day non-profit
and hybrid ownership forms still dominate most markets \citep{cull2009,conning2011}. The model can also help us think about recent episodes where rising competition and "commercialization" appear to have been associated with periods of rising refinancing, multiple borrowing and indebtedness and rising complaints of insufficient consumer protection. In some instances this has led to financial crashes and strong political backlash as in the case of the 2010 microfinance crisis in Andhra Pradesh, India, or the 2008 sub-prime loan financial crisis.

Our model provides a consistent framework for further analysis of such issues. As such it complements \citet{bubb2013} which also builds a model of endogenous
firm ownership structure as a form of consumer protection, but the
underlying behavioral stories are quite different. In their analysis
firms hide non-contractible penalties in loan contracts and opportunistically
charge such fees on a mix of suspecting and unsuspecting risk-neutral customers. Theirs is not so much a model of conflicting
selves as a model of hidden penalties that requires a population of
exploitable naive borrowers to produce an inefficiency. In contrast
our model is built upon a contract-renegotiation problem that survives even with sophisticated customers and full information. By focusing (primarily) on sophisticated consumers and setting aside information asymmetries, we aim to emphasize the implications and importance of renegotiation concerns.

\section{The model: setup}

We work with a three-period consumption smoothing model for a present-biased
consumer with quasi-hyperbolic preferences that allows for saving
(repayment) or borrowing (dissaving) in each period. In any period
$t\in\left\{ 0,1,2\right\} $, the consumer's instantaneous utility
from consumption level $c_{t}$ is given by a CRRA function defined
over all non-negative consumption: 
\begin{equation}
u\left(c_{t}\right)\equiv\frac{c_{t}^{1-\rho}}{1-\rho}
\end{equation}
with some $\rho>0$ as the coefficient of relative risk aversion.\footnote{When $\rho=1$ the function becomes $u(c_{t})=ln(c_{t})$.}

We model the consumer "as a sequence of temporal selves ... indexed
by their respective periods of control over the consumption decision"
\citep[p.451]{laibson1997}. Given a consumption stream $C_{t}\equiv\left(c_{t},..,c_{2}\right)$,
the period-$t$ self's discounted utility is: 
\begin{equation}
U_{t}\left(C_{t}\right)\equiv u\left(c_{t}\right)+\beta\sum\limits _{i=t+1}^{2}\delta^{i-t}u\left(c_{i}\right)\label{eq:obj}
\end{equation}
This describes quasi-hyperbolic preferences, with a standard exponential
discount factor $\delta\in(0,1]$ and a hyperbolic discount factor
$\beta\in(0,1)$. In any period $t$, the individual (henceforth referred
to as the ``$t$-self'') discounts the entire future stream of utilities
by $\beta$. As a result, when faced with any tradeoff between periods $t$ and $t+x$, the $t$-self places greater relative
weight on period-$t$ consumption than her earlier selves would have. 

The consumer could be sophisticated (her time-inconsistency
is common knowledge across all $t$-selves) or naive (she believes
her future selves to be exponential discounters with a discount factor
of $\delta$).\footnote{We ignore the possibility of partial naivete \citep[see][]{odonoghue2001}.}

The Zero-self begins with an endowment of claims to an arbitrary positive
income stream over the three periods, $Y_{0}\equiv\left(y_{0},y_{1},y_{2}\right)$.
Her objective is to rearrange this into a preferable consumption stream
$C_{0}$ to maximize $U_{0}(C_{0})$ in (\ref{eq:obj}) using whatever
financial contracting or other saving/borrowing strategies may
be available.

In the absence of access to the financing and commitment services
offered by a bank, the consumer can only achieve an "autarky" consumption
stream which delivers a corresponding autarky utility denoted $U_{0}^{A}$.
The simplest assumption is that this autarky consumption stream corresponds
to the endowment income stream. More realistically, the autarky consumption
stream is what might be achieved via the more limited financing and
commitment services available through informal banking or self-commitment
strategies.

Section \ref{sec-FCC} describes the benchmark optimal consumption
smoothing stream $C_{0}^{F}$ and associated utility level $U_{0}^{F}$
that Zero-self could achieve if she had perfect access to borrowing
and saving at competitive interest rates with all the commitment
required to make sure the contract is not renegotiated. There are
many reasons why in practice autarky consumption plans might fall
short of this optimum. For example, if the consumer's income is back-heavy,
borrowing constraints might mean she must consume income as it arrives.
If her income is front-heavy she may be able to construct a somewhat
smoothed consumption stream but there may be technological restrictions
to saving that place the return to savings well below the market rate
\textendash{} the insecurity of storing cash at home being one obvious
explanation. More pivotal to our analysis, however, is that even with
access to perfectly secure savings or borrowing, a consumer with time-inconsistent
preferences cannot trust her later selves to follow her optimal consumption
path. While remaining deliberately agnostic about autarky technologies,
the rest of the paper focuses on the reasonable and interesting case
$U_{0}^{A}<U_{0}^{F}$ where there are potential gains to financial
contracting with a new intermediary.

The consumer will have the option to contract with one or many risk-neutral
banks, depending on whether the period 0 market structure is monopolized
or competitive. Each bank can access funds at interest rate opportunity
cost $r$. At this market interest rate, the present value of the
consumer's income stream is: 
\begin{equation}
y\equiv\sum\limits _{i=0}^{2}\frac{y_{i}}{\left(1+r\right)^{i-t}}
\end{equation}

A period 0 financial contract allows the consumer to exchange income
stream $Y_{0}$ for a new smoother consumption path $C_{0}$. A bank
will participate if and only if it can expect to earn non-negative
profits $\Pi_{0}(C_{0};Y_{0})$, where profits are defined as:

\begin{equation}
\Pi_{t}(C_{t};Y_{t})\equiv\sum\limits _{i=t}^{2}\frac{\left(y_{i}-c_{i}\right)}{\left(1+r\right)^{i-t}}\label{eq:profit}
\end{equation}
The contract may involve borrowing (or dissaving) or savings (or
paying down debt) in period $t$ depending on whether $(c_{t}>y_{t})$
or $(c_{t}<y_{t})$, respectively. We begin by assuming contracts
can only be initiated in period 0.\footnote{This assumption helps simplify the consumer's participation constraint without directly affecting the analysis of the renegotiation problem. \citet{basu2020} analyzes the case where commitment contracts can be initiated in any period. We discuss this further in the Conclusion.} Contracts may however be renegotiated by the consumer and the original
bank or possibly a new one in period 1. If this happens, we assume
the bank incurs a non-monetary cost, $\kappa\geq0$.\footnote{We discuss the source and nature of such costs in depth in section
\ref{nonprofits}. The bank could incur monetary costs in addition
to the non-pecuniary ones but we assume these to be 0 as they can
be netted out and do not affect the analysis in any important way.} This could be interpreted as a concern for the consumer's
well-being, own reputation, or pure transaction costs.

In each contracting scenario the sophisticated consumer's Zero-self
has a bias for present consumption but wants to smooth future consumption
across periods 1 and 2. She correctly anticipates that her One-self
will have a change of preferences that will lead her to want to "raid
savings" and/or take on new debt to drive up period 1 consumption
at the expense of period 2 consumption, thereby undoing Zero-self's
early intent to balance consumption across the two periods. In every
case below, a sophisticated Zero-self chooses a contract anticipating One-self's and
bank reactions, possibly limited by the bank's exogenously or endogenously
enforced commitment to not renegotiate with One-self. In contrast,
the naive consumer does not anticipate possible renegotiation (an
error the bank may choose to exploit).

\section{Full-smoothing commitment contracts }

We first characterize optimal consumption-smoothing contracts when
the consumer can perfectly and costlessly bind their latter selves
to not renegotiate contracts with the same bank or
other banks. We do this for the case of competition and monopoly.

\subsection{ Competitive Contracts}

\label{sec-FCC} A consumer with time-inconsistent preferences cannot
trust her latter selves to stick to her preferred consumption plans.
Similar to a Stackelberg-leader in a Cournot game, Zero-self's strategic
saving/borrowing choices will be affected by her anticipation of One-self's
and the banks' best renegotiation response. 

If banks can be assumed to credibly and costlessly commit to never
renegotiate\footnote{ For now this also means such contracts are "exclusive" in
that in later periods no new bank can enter to "buy out"
or renegotiate a contract or, equivalently, that they too are dissuaded
from it by credible penalties. } then the sophisticated consumer's self-control problem is removed.
Competition for period 0 contracts ensures Zero-self will, in
effect, choose a preferred contract that commits her One- and Two-selves
to follow the chosen consumption plan. This is a standard utility
maximization problem subject to an inter-temporal budget constraint
(i.e. the financial intermediary's zero-profit condition). Zero-self
chooses $C_{0}$ to solve: 
\begin{align}
\max_{C_{0}} & \ U_{0}(C_{0})\label{eq:cobj0}\\
\text{s.t.} & \ \Pi_{0}(C_{0};Y_{0})\geq0\label{eq:BPC0}
\end{align}


The familiar first-order necessary conditions are: 
\begin{equation}
u'\left(c_{0}\right)=\beta\delta(1+r)u'\left(c_{1}\right)=\beta\delta^{2}(1+r)^{2}u'\left(c_{2}\right)
\end{equation}

An increase or decrease to the term $\delta(1+r)$ essentially `tilts'
the consumption profile to generally rising or falling over time as
$\delta\gtreqless\frac{1}{1+r}$. As this across-the-board tilt will
not alter key tradeoffs of interest (unlike the degree of present-bias,
$\beta$, which does) we shall impose the assumption that
$\delta=\frac{1}{1+r}=1$ for the remainder of the analysis. This
is without loss of generality and greatly unclutters the notation. The
simplified first-order conditions are: 
\begin{equation}
u'\left(c_{0}\right)=\beta u'\left(c_{1}\right)=\beta u'\left(c_{2}\right)\label{eq:FOC_comp}
\end{equation}
The binding bank zero profit constraint and first-order conditions
allow us to solve for the competitive full-smoothing commitment
contract $C_{0}^{F}$. For the CRRA case, a closed form solution for $C_{0}^{F}$ is
easily found (\ref{eq:c-f})\footnote{All CRRA derivations and closed-form solutions are in the appendix.}
and the FOCs can be written: 
\begin{equation}
c_{1}^{F}=c_{2}^{F}=\beta^{\frac{1}{\rho}}c_{0}^{F}
\end{equation}
Zero-self indulges her present bias (by tilting consumption toward
herself) and then allocates remaining resources  evenly
across the remaining two periods. More generally, if viewed over a longer horizon, a full-smoothing commitment contract ensures steady, equal, consumption across all future periods (permitting only the initial signatory, the Zero-self, some indulgence).


Consider the simple example where $\beta=0.5$, $\rho=1$ and endowment
income has present value $\sum y_{t}=300$. Zero-self's preferred
commitment contract will be $C_{0}^{F}=(150,75,75)$. If the total
income arrives evenly across periods as $Y_{0}=(100,100,100)$ then
this consumption plan would imply borrowing of $c_{0}^{F}-y_{0}=50$
in period 0 to be repaid in equal installments of 25 in periods 1
and 2. Had the stream instead been $Y_{0}=(200,50,50)$ the consumer
would save 50 in period 0 to raise consumption by 25 in each of periods
1 and 2. We'll return to these simple numerical examples below to illustrate
why, when commitment becomes costly and imperfect, One-self may carry
"too much debt" or "not save enough" relative to Zero's preferred
choices.\footnote{These parameter values are chosen for expositional purposes. In particular
$\rho=1$ implies that period 0 consumption will be the same with
or without commitment but the analysis is easily adapted to other
values.}

\subsection{Monopoly Contracts}

\label{sec:own}

When the bank has monopoly power in period 0 the optimal contract
will maximize bank profits subject to a consumer participation constraint:
\begin{align}
\max_{C_{0}} & \ \Pi_{0}\left(C_{0};Y_{0}\right)\label{eq:monop-obj}\\
\text{s.t.} & \ U_{0}\left(C_{0}\right)\geq U_{0}^{A}\label{eq:CPC0}
\end{align}
The first-order tangency conditions are again given by expression
(\ref{eq:FOC_comp}). Substituting these into Zero-self's participation
constraint, which must bind at a monopoly optimum, we can solve
for the optimal contract $C_{0}^{mF}$ and corresponding bank profits
$\Pi_{0}\left(C_{0}^{mF};Y_{0}\right)$. Closed form solutions for
the CRRA case appear as appendix equations (\ref{eq:c-mf}) and (\ref{eq:pi-mf}).
Consumption $C_{0}^{mF}$ rises and profits fall with the consumer's
reservation autarky utility $U_{0}^{A}$.

Conceptually, the equilibrium contract under competition is found
at the tangency between the highest iso-utility surface just touching
the budget hyper-plane. Under monopoly, the optimum will
be at the tangency point where the highest iso-profit plane just touches
the iso-utility surface associated with Zero-self's reservation utility.
Since the monopolist bank fully captures the gains to trade, consumption
in each period will be lower than under competition. 

\section{The Renegotiation Problem}

\label{renegotiation}

Now to  the heart of the paper: when is commitment credible,
how is it sustained, and at what cost? At issue is the fact that One-self
always prefers higher period 1 consumption than what Zero-self wants
to build into a contract, so there may be tempting gains to trade
from breaking earlier contract commitments. The credibility of the
bank's commitment must in turn rest on the threat of a sufficiently
costly punishment $\kappa$ to deter it from such renegotiation.
Below we first derive the minimum deterrent punishment required to
sustain full-smoothing commitment and then study contract adaptations when available deterrents fall short of this threshold.

Figure \ref{fig:c1c2} depicts the renegotiation problem for the case where renegotiation costs are
set to $\kappa=0$, which is to say where One-self and a bank can rewrite
any contract with no penalty.\footnote{Figure 1 is drawn for  $\beta=0.4$, $\rho=08$, $\kappa=0$, and $y= [100, 100, 100]$.  Values were chosen to accentuate curvature and keep the figure uncluttered.  All figures can be replicated and changed in response to parameter values using the online interactive appendix.} Assume \textendash{} just for the sake
of argument now \textendash{} that the consumer had (naively as it
will turn out) accepted a full-smoothing commitment contract $C_{0}^{F}$
in period zero (or $C_{0}^{mF}$ in the monopoly case). The associated
continuation contract is depicted at point $F$ in the $c_{1}-c_{2}$
plane. This contract satisfies Zero-self's first-order condition $u'(c_{1})=u'(c_{2})$
as indicated by the fact that Zero-self's indifference curve is tangent
to the bank's iso-profit line, or equivalently that \(F\) lies on the \(c_2=c_1\) ray. Since One-self discounts period 2 utility
more heavily, this bundle provides too much period 2 consumption from her perspective ($u'(c_{1})>\beta u'(c_{2})$). There are mutual gains-to-trade that could be shared by recontracting
from $F$ to any new tangency point between points \(R\) and \(P\) along the $c_{2}=\beta^{\frac{1}{\rho}}c_{1}$
ray where One-self's first-order conditions are met. Point
$R$ is the renegotiated contract least favorable to One-self
(chosen if the bank could act as monopolist in period 1) and point
$P$ is the most favorable to One-self
(chosen under competitive renegotiation).


Being a sophisticate, Zero-self of course anticipates this problem and will only agree to contracts that satisfy a no-renegotiation constraint
to deter the bank(s) and her future self from any such harmful renegotiations. The addition of a new binding constraint can however only reduce the feasible contract space, hence reducing
consumer welfare and/or bank profits and trade. When the market for
period 0 contracts is competitive, consumer welfare will be reduced.
If instead the market is monopolized bank profits will suffer as the
bank can no longer offer the least cost (most profitable) smoothing
contract.

\begin{figure}[p]
  \includegraphics[scale=0.7]{Figure1.pdf}
  \caption{Optimal competitive contract and renegotiation threat}
  \label{fig:c1c2}
\end{figure}


\subsection{Renegotiated contracts}

To derive a no-renegotiation constraint we must understand the
terms of renegotiated contracts even if, in equilibrium, no renegotiations
will take place. Consider a contract $C_{0}^{0}=(c_{0}^{0},c_{1}^{0},c_{2}^{0})$
and the period 1 subgame determined by its associated continuation
contract $C_{1}^{0}=(c_{1}^{0},c_{2}^{0})$. A renegotiation takes
place when One-self and a bank agree to replace continuation contract
$(c_{1}^{0},c_{2}^{0})$ by a new contract $\left(c_{1}^{1},c_{2}^{1}\right)$.


First consider the case when period 1 banks compete to replace contract
$C_{1}^{0}$ with renegotiated contract $C_{1}^{1}(C_{1}^{0})$. This contract is the solution to: 
\begin{align}
\max_{C_{1}} & \ U_{1}(C_{1})\\
s.t. & \ \Pi_{1}(C_{1};C_{1}^{0})\ge\kappa\label{eq:PiGain}
\end{align}
{}where the bank participation constraint (\ref{eq:PiGain}) can be stated
as $(c_{1}^{0}+c_{2}^{0})-(c_{1}+c_{2})\ge\kappa$. To entice a bank
to participate the renegotiated contract must reduce contract expenses (increase bank profits) by an amount that equals or exceeds the
renegotiation cost. Competition insures this constraint exactly
binds. We can solve for an
interior competitive renegotiation contract $C_{1}^{1}(C_{1}^{0})$ using the first-order condition $u'(c_{1}^{1})=\beta u'(c_{2}^{1})$
and binding condition (\ref{eq:PiGain}).\footnote{ For CRRA utility, the contract will be renegotiated to $c_{1}=\frac{c_{1}^{0}+c_{2}^{0}-\kappa}{1+\beta^{\frac{1}{\rho}}}$
and $c_{2}=\beta^{\frac{1}{\rho}}c_{1}$. } For example, with zero renegotiation costs ($\kappa=0$) contract
$F$ in Figure \ref{fig:c1c2} would be renegotiated to $P$. For
positive $\kappa$ (but less than $\bar{\kappa}$ in the figure) the
consumer will surrender just enough surplus to the bank as to get
them to participate, resulting in a contract that lies between $P$
and $R$.

If a bank is a monopolist in period 1, and $\kappa$ is not so high
as to make renegotiation infeasible, the bank's preferred renegotiated contract ($C_{1}^{1m}(C_{1}^{0})$) would solve:

\begin{align}
\max_{C_{1}}\ & \Pi_{1}(C_{1};C_{1}^{0})-\kappa\\
\text{s.t.} & \ U(C_{1})\geq U(C_{1}^{0})\label{eq:ugain}
\end{align}
In Figure \ref{fig:c1c2} the monopolist would renegotiate contract
$F$ to a point \textit{just} above $R$ to entice One-self
to participate, while it captures (practically) all the gains to renegotiation. Appendix equation (\ref{eq:m-r}) shows the monopolist's
preferred renegotiated contract for any continuation contract
$C_{1}^{0}$.


For future reference, note that $\bar{\kappa}$ represents the maximum monetary gain that could be earned from renegotiation. We formalize this term later in the section.

 
\begin{figure}[p]
\includegraphics[scale=0.7]{Figure2.pdf}\caption{The no-renegotiation constraint}
\label{fig:renegproof} 
\end{figure}


\subsection{The `no-renegotiation' condition}

\label{sec-no-reneg-cond}

When will a contract \textit{not} be renegotiated in period 1? Assuming
a tie-breaking rule in favor of Zero-self's preferences, depending
on whether the market structure in period 1 is monopolized or competitive,
respectively, the conditions for no renegotiation in period 1 can be described by:
\begin{align}
U_{1}\left(C_{1}^{1}\left(C_{1}^{0}\right)\right) & \leq U_{1}\left(C_{1}^{0}\right)\label{eq:no-reg-comp}\\
\Pi_{1}\left(C_{1}^{1m}\left(C_{1}^{0}\right);C_{1}^{0}\right) & \leq\kappa\label{eq:no-reg-monop}
\end{align}

In fact these two conditions express the same thing: a period 0 contract will be renegotiation-proof if and only if it is \textit{not} possible in period 1 for a bank and One-self to agree to a new contract that simultaneously (a) leaves One-self with at least as much discounted utility as the original contract, and (b) generates a profit gain of at least $\kappa$ to the bank. In short, the contract will be renegotiation-proof as long as renegotiation costs are large enough to exhaust any potential gains to trade between the two parties. These requirements can be expressed as a single no-renegotiation condition. For the CRRA case:

\begin{align}
u(c_{1}^{0})+\beta u(c_{2}^{0})\ge u\left(\frac{c_{1}^{0}+c_{2}^{0}-\kappa}{1+\beta^{\frac{1}{\rho}}}\right)(1+\beta^{\frac{1}{\rho}})\label{ineq:noreg}
\end{align}

The determination of renegotiation-proof continuation contracts in $c_{1}-c_{2}$ space is illustrated in Figure \ref{fig:renegproof}.\footnote{Figure 2 parameters: $\kappa=8$, $\beta=0.4$, $\rho=0.6$, and $y=[100, 100, 100]$)} Suppose the exogenous cost of renegotiation, $\kappa$, is smaller than $\bar{\kappa}$ (the maximum gains from renegotiation in Figure \ref{fig:c1c2}). Then, the full-smoothing contract, $F$, would be renegotiated because the potential gains from renegotiation exceed the penalty.
So the Zero-self and the bank must find a second-best renegotiation-proof contract. This involves sufficiently accommodating One-self's preferences (transferring consumption from period 2 to period 1) to make renegotiation less attractive. One such possibility is the continuation contract at point $C$. One-self would renegotiate this to any contract lying above their indifference
curve running through $C$. The bank, however, will only agree to a proposed renegotiation if it can lower contract costs to below existing costs net of the renegotiation cost $\kappa.$ In diagram terms, the
bank will agree only if the renegotiated contract falls \textit{below} the lower of the drawn isocost lines with period 1 cost $c_{1}^{0}+c_{2}^{0}-\kappa$.
Contract $C$ is (barely) renegotiation-proof because even the most generous renegotiation that One-self can offer the bank (contract $R$) falls just short of raising the banks profits by enough to compensate for renegotiation costs.

Given $\kappa$, the no-renegotiation constraint restricts the set of feasible contracts to a subset of the $c_{1}-c_{2}$ space. The boundaries of this set are described by a binding equation (\ref{ineq:noreg}).\footnote{From  an examination of the no-renegotiation constraint
(\ref{ineq:noreg}) it is clear there are, in fact, two non-linear boundaries that satisfy this condition. However, given any curvature in \(u(c), \text{ } \)the boundary that delivers higher period 1 consumption must always bind first, so this is the one we focus on.
When $\kappa=0$ the region coincides with the $c_{2}=\beta^{\frac{1}{\rho}}$
line.} When the no-renegotiation
constraint binds in our profit or utility maximization problem, the
bank must offer a contract along this boundary.\footnote{Recall that by assumption ties are broken in favor of Zero-self's
preferences so no-renegotiation is weakly preferred here. } 


\subsection{Threshold external renegotiation costs }

What is the minimum renegotiation cost sufficient to deter the renegotiation
of a full-smoothing commitment contract? This can be found by setting $c_{1}^{0}=c_{1}^{F}=c_{2}^{0}$
in the no-renegotiation condition (\ref{ineq:noreg}) and solving
for $\kappa$. A competitive full-smoothing commitment contract will
survive if and only if 
\begin{equation}
\kappa\geq\bar{\kappa}\equiv c_{1}^{F}\cdot\Upsilon\label{eq:kbar}
\end{equation}
while a monopolistic full-smoothing commitment contract will survive
if and only if: 
\begin{equation}
\kappa\geq\bar{\kappa}^{m}\equiv c_{1}^{mF}\cdot\Upsilon\label{eq:kbarM}
\end{equation}
where $\Upsilon$ is the constant in (\ref{eq:upsilon}).

Here $\bar{\kappa}$ and $\bar{\kappa}^{m}$ are the threshold minimum
renegotiation costs required to deter the renegotiation of the first-best
full-smoothing commitment contract. The greater the consumption
levels in the contract (the greater the scope for profitable
contract rearrangements in period 1), the more costly it becomes to
deter renegotiation.
Under competition $c_{1}^{F}$ is independent of autarky utility (given
a fixed value of $y$) so $\bar{\kappa}$ does not depend on how close
or far from optimal consumption smoothing the consumer is in autarky.
With monopoly in period 0 the threshold $\bar{\kappa}^{m}$ which
rises linearly with $c_{1}^{mF}$ will be increasing in autarky utility
$U_{0}^{A}$ (see \ref{eq:c-mf}). Since $c_{1}^{mF}<c_{1}^{F}$ for
any initial $Y_{0}$ we must also always have $\bar{\kappa}^{m}<\bar{\kappa}$.
Proposition 1 summarizes: 
\begin{prop}
\label{Prop:full-commit} Given renegotiation cost thresholds  $\bar{\kappa}$
and $\bar{\kappa}^{m}$ as defined by \ref{eq:kbar} and
\ref{eq:kbarM}: 
\begin{enumerate} [label=\alph*)]
\item The competitive full-smoothing commitment contract survives
if and only if $\kappa\geq$$\bar{\kappa}$. 
\item The monopolistic full-smoothing commitment contract survives
if and only if $\kappa\geq$$\bar{\kappa}^{m}$ with $\bar{k}^{m}$
strictly rising in the consumer's autarky utility. 
\item $\bar{\kappa}^{m}<\bar{\kappa}$. 
\end{enumerate}
\end{prop}
An implication is that under
monopoly, consumers with better autarky options are less likely to
get full-smoothing commitment contracts. A
consumer with higher autarky utility must be offered higher consumption
by the monopolist, and the no-renegotiation condition is harder to
satisfy at higher levels of consumption. This  dampens the
advantages of improved outside options for sophisticated hyperbolic
discounters contracting with monopoly banks.

A further implication is that monopoly is relatively better than competition at delivering full-smoothing
commitment contracts ($\bar{\kappa}^{m}<\bar{\kappa}$),
but this is not because monopolists are inherently better at committing. Instead,
this result follows from the fact that having at the outset extracted
surplus by offering the consumer a contract with the lowest possible
consumption, there is less surplus left to be captured
via renegotiation in period 1.\footnote{There may be other reasons outside of this model that make monopolists
better at committing (i.e. having a higher $\kappa$). Our point is
that this is not necessary for monopolists to offer better smoothing
in more circumstances than competitive firms.}

\section{Imperfect-Smoothing Commitment Contracts }

\label{sec:imperfectK}

When bank renegotiation costs are not high enough to sustain full-smoothing
commitment contracts, that is where $\kappa<\bar{\kappa}$ under competition
or $\kappa<\bar{\kappa}^{m}$ under monopoly, renegotiation-proofness
will require contract distortions which we now characterize.

A bank that contracts with naifs will capitalize on the consumer's
failure to anticipate harmful future renegotiations (Section 5.3).
A sophisticated consumer is wise to the problem and will only agree
to renegotiation-proof contracts, as will the bank (Sections 5.1 and 5.2).\footnote{Observe that any contract that involves renegotiation could be replaced by a renegotiation-proof contract that delivers the same consumption path without incurring the renegotiation penalty.} In the absence
of sufficiently high external renegotiation penalties however the
parties will resort to additional endogenous enforcement mechanisms
by shifting the terms of continuation contracts closer to One-self's
preferred choices as a strategy to reduce the gains to renegotiation. This can only harm bank profits relative to full-smoothing since it raises the contract cost of
keeping the consumer at their reservation utility.
These "imperfect-smoothing commitment" contracts are
still technically "full commitment" contracts in the sense that renegotiation
is avoided in equilibrium but they generally provide less than perfect
or efficient consumption smoothing from Zero-self's perspective compared
to contracts with stronger external enforcement penalties. As we
shall later see, banks may be willing to spend to improve externally imposed renegotiation
penalties.

For expositional convenience, we first discuss the monopoly case.

\subsection{Monopoly}

When the market for period 0 contracts is monopolized the bank will
want to maximize multi-period profits subject to Zero-self's participation
and the no-renegotiation constraint:

\begin{align}
\max_{C_{0}} & \ \Pi_{0}\left(C_{0};Y_{0}\right)\\
s.t. & \ U_{0}\left(C_{0}\right)\geq U_{0}^{A}\label{eq:pc-m}\\
 & \ \Pi_{1}\left(C_{1}^{1}\left(C_{1}\right);C_{1}\right)\leq\kappa\label{eq:rpc-m}
\end{align}


The bank wants to search for the most profitable renegotiation-proof
contract that lies on Zero-self's participation constraint (\ref{eq:pc-m}).
Consider a candidate level of period 0 consumption $c_{0}^{0}$. The
associated continuation contract $C_{1}^{0}$ must lie along Zero-self's
autarky utility surface which can be projected as indifference curve
$\beta\left[u(c_{1}^{0})+u(c_{2}^{0})\right]=U_{0}^{A}-u(c_{0}^{0})$
in $c_{1}-c_{2}$ space. Let this be represented by Zero-self's indifference
curve in Figure \ref{fig:renegproof}. Note this indifference curve
shifts down or up as we increase or decrease $c_{0}^{0}$, which for
the moment we take as given. Many continuation contracts are both
renegotiation-proof and satisfy Zero-self's participation (all in
the area above the indifference curve and below the no-renegotiation
boundary) but the most profitable amongst these will be at point $C$
in Figure \ref{fig:renegproof} at the intersection of the two constraints.
This gives us the optimal renegotiation-proof continuation contract
$C_{1}^{m}(c_{0}^{0})$ from any $c_{0}^{0}$. The monopolist's optimal
contract is then determined by choosing over $c_{0}^{0}$.

\subsubsection{Properties of the contract}

\label{sec_contract_properties}

The renegotiation-proof contract can be explicitly derived for the
CRRA case of $\kappa=0$ (\ref{eq:zerokappa-monop}). For,
$0<\kappa<\bar{\kappa}^{m}$, the contract cannot be derived in closed
form because there are two points where the participation constraint
and no-renegotiation constraint are satisfied with equality (at the
upper and lower boundaries of the set of renegotiation-proof contracts).
However, the key properties of the equilibrium contract can be established
and it can be easily solved for numerically.
\begin{prop}
Suppose $\kappa<\bar{\kappa}^{m}$ and the consumer is sophisticated.
Under monopoly, the profit-maximizing renegotiation-proof contract
($C_{0}^{mP}$) has the following properties:
\begin{enumerate} [label=\alph*)]
\item $\Pi_{0}\left(C_{0}^{mP};Y_{0}\right)<\Pi_{0}\left(C_{0}^{mF};Y_{0}\right)$

\item $c_{0}^{mP}>c_{0}^{mF}$ 
\end{enumerate}
\end{prop}
Proposition 2 compares the renegotiation-proof contract to the full-smoothing
commitment contract when the no-renegotiation constraint binds.
First, bank profits will be lower than under full-smoothing commitment.
The bank wishes it could promise to not renegotiate but it cannot
make such a promise credible without giving up some profits. The monopolist
would be better off with higher external renegotiation penalties since
in equilibrium renegotiation does not take place.\footnote{A related observation is that the bank will prefer not to contract
with individuals who have minimal smoothing needs -- for individuals
whose autarky utility is close enough to $U_{0}^{F}$, the bank would
make negative profits under the best renegotiation-proof contract,
so there will be lost trade.}

The second statement of the proposition is about the terms of the
contract \textendash{} when full-smoothing commitment is not feasible,
the renegotiation-proof contract will involve higher consumption in
period 0 compared to
full-smoothing commitment. The following is a sketch of the argument
(the proof in the appendix uses some additional notation for logical
clarity).

Consider a potential imperfect-smoothing commitment contract that happens to deliver the same period 0 consumption as under full-smoothing ($c_{0}=c_{0}^{mF}$). Now, we ask: could Zero-self's utility be raised by moving consumption to or from the future? In other words, does she gain more from an additional dollar in period 0, or from an additional dollar allocated across periods 1 and 2 ("the future")? 

It turns out that, under this hypothetical contract, from Zero-self's perspective the marginal utility of future consumption is relatively low. This is for two reasons. First, if period 0 consumption is the same as under full-smoothing, future consumption must be substantially higher than under full-smoothing to simultaneously satisfy Zero-self's participation constraint and the no-renegotiation constraint (\ref{eq:s-compare}). Second, to continue satisfying the no-renegotiation constraint, every additional dollar allocated to the future must be split increasingly in period 1's favor (\ref{eq:dalpha-ds}). Together, these factors ensure that the returns to future consumption are small -- the bank could raise its profits by increasing period 0 consumption and lowering future consumption. The equilibrium contract therefore involves a larger loan or less savings than under a full-smoothing commitment contract. 

In effect, what might look like a bank feeding the consumer's temptation through greater immediate consumption, is in fact a strategy to limit renegotiation possibilities by transferring consumption
away from the future.

\subsection{Competition}

When the market for period 0 contracts is competitive the optimal
contract solves:

\begin{align}
\max_{C_{0}} & \ U_{0}\left(C_{0}\right)\label{eq:comp-max}\\
\text{s.t.} & \ \Pi_{0}\left(C_{0};Y_{0}\right)\geq0\label{eq:comp-pc}\\
 & \ \Pi_{1}\left(C_{1}^{1}\left(C_{1}\right);C_{1}\right)\leq\kappa\label{eq:no-reg-competition}
\end{align}

As noted in section \ref{sec-no-reneg-cond}, the no-renegotiation
constraint (\ref{eq:no-reg-competition}) assures that gains-to-trade from
renegotiation fall short of bank renegotiation costs. Even if new
banks could enter in period 1 to offer part or or all of the surplus
from renegotiation to One-self in period 1, the constraint deters renegotiation
as long as those banks also face renegotiation cost $\kappa$.\footnote{If other banks face lower renegotiation costs, the no-renegotiation constraint simply gets tighter. Later we discuss the empirically relevant cases where competing banks
might enter in period 1 and offer to renegotiate at lower or zero
renegotiation cost).}

We can reuse Figure \ref{fig:renegproof} to interpret the contract
design. Zero-self wants to search for the most profitable renegotiation-proof
contract that lies on the bank's participation constraint. Suppose Zero-self has chosen a candidate period 0 level
of consumption $c_{0}^{0}$. Let the outer isoprofit line represent the resulting zero-profit line for the bank in  $c_{1}-c_{2}$ space. To be part of an optimum renegotiation-proof
contract Zero-self must ensure that the continuation contract is renegotiation
proof and satisfies the bank's zero-profit constraint $c_{1}^{0}+c_{2}^{0}=y-c_{0}^{0}$.
Many continuation contracts are both renegotiation-proof and satisfy
the zero-profit constraint (all below the zero-profit line and
within the no-renegotiation set) but the most-preferred by Zero-self
will be at point $C$ in Figure \ref{fig:renegproof} at the intersection
of the two constraints.\footnote{Note that while we are reusing Figure 2 to describe both the monopoly
and the competitive contract design problem because, conceptually,
they are very similar, optimal consumption levels will be generally
higher under competition so the point $C$ is not the same in both
cases.} This gives us the optimal renegotiation-proof continuation contract
$C_{1}^ {}(c_{0}^{0})$ from any $c_{0}^{0}$. Zero-self's optimal
contract is then determined by backward induction, choosing over $c_{0}^{0}$.

In the special case of perfect competition with costless renegotiation
($\kappa=0$) there will be a unique solution and a closed form. In
Figure \ref{fig:renegproof} think of how $C$ slides down the bank's
zero-profit line as $\kappa$ shrinks until we get to a point where
One-self's indifference curve is tangent to the zero-profit line.
This continuation contract is "renegotiation-proof" only in the very
narrow sense that it won't be renegotiated because it already delivers
One-self's preferred consumption choice. This contract is explicitly
solved in expression (\ref{eq:zerokappa-comp}).

Consider a simple example: with $\beta=0.5$
and $\rho=1$ and $\kappa=0$ the best available competitive contract
$C_{0}^{P}=(150,100,50)$ offers considerably less consumption smoothing
in later periods compared to the benchmark full-smoothing $C_{0}^{F}=(150,75,75)$.
Supose the initial income stream were arranged as $Y_{0}=(100,100,100).$
We  could then interpret the absence of commitment case ($\kappa=0$) as
rolling over of debt that Zero would have preferred to have
seen evenly repaid in periods 1 and 2. Instead the entire burden of the debt that Zero took out in period
0 falls due in period 2. Had the income stream
instead started as $Y_{0}=(200,50,50)$ then we might interpret the consumer
in period one as "raiding savings" in period 1 that, with commitment, Zero-self
would have preferred protected for period 2 consumption.

\subsubsection{Properties of the contract}

Let $C_{0}^{P}$ denote the solution to the maximization
problem described by \ref{eq:comp-max}, \ref{eq:comp-pc} and \ref{eq:no-reg-competition}. 
\begin{prop}
Suppose $\kappa<\bar{\kappa}$ and the consumer is sophisticated.
The competitive renegotiation-proof contract that
maximizes Zero-self's discounted utility ($C_{0}^{P}$) has the following
properties:
\begin{enumerate} [label=\alph*)]
\item $U_{0}\left(C_{0}^{P}\right)<U_{0}\left(C_{0}^{F}\right)$

\item The relationship between $c_{0}^{P}$ and $c_{0}^{F}$ is ambiguous.
There is some $\hat{\rho}$ such that: if $\rho\leq\hat{\rho}$, then
$c_{0}^{P}>c_{0}^{F}$; if $\rho>\hat{\rho}$, then there are parameter
values under which $c_{0}^{P}<c_{0}^{F}$. 
\end{enumerate}
\end{prop}
The first statement is straightforward: Since $\kappa<\bar{\kappa}$
means the new renegotiation-proofness constraint (\ref{eq:no-reg-competition})
binds, full-smoothing smoothing cannot be achieved and the consumer's
welfare must be lower than under the first-best contract.

The second statement regards whether period 0 consumption  (and hence period 0 net borrowing) will be higher or lower compared to the full-smoothing
commitment contract. The proposition is that this depends on the intertemporal elasticity of substitution $\frac{1}{\rho}$.

Again, let us imagine a candidate contract $C_{0}$ that
involves the same period 0 consumption as under full-smoothing commitment,
so that $c_{0}=c_{0}^{F}$. Unlike under monopoly, total future consumption under this contract will be the same as (as opposed to higher than) under full-smoothing (because the contract satisfies the bank's zero-profit constraint rather than the consumer's participation constraint). However, this future consumption will be split in period 1's favor.

Could Zero-self do better by moving a dollar to or from the future? If the utility function is relatively linear (low $\rho$), then the imbalanced future allocation induced by the no-renegotiation constraint lowers the marginal utility of future consumption. The opposite happens if the utility function is highly concave (high $\rho$).\footnote{The  cutoff value $\hat{\rho}$ is somewhat
complicated, as the marginal utility of future consumption depends not just on the allocation induced by the no-renegotiation constraint, but how the allocation changes at the margin.}

So, under competition, a binding no-renegotiation constraint can
change the contract in either direction: a larger period zero loan (less saved)
or a smaller loan (more saved). This can be seen clearly for the case of $\kappa=0$ where from the closed form expression for \(c_0^P\) (\ref{eq:zerokappa-comp}) we can show  \(c_{0}^{P} \gtrless c_{0}^{F} \) as \(\rho \lessgtr 1\).

Period 2 consumption however always
falls relative to full-smoothing commitment, even in the
cases when Zero-self saves more/borrows less. In fact for CRRA utility
the adjustment of period 0 consumption (in response to the renegotiation threat) is always relatively small while the
adjustment to period 1 and period 2 consumption is proportionately much
larger.\footnote{To illustrate, with $\kappa=0$ at no point does period 0 consumption
rise or fall by more than six percent for any value $\rho\in(0,\infty)$
and $\beta\in(0,1)$ but at reasonable parameter values such as $\rho=0.5$
and $\beta=0.5,$  period 1 consumption
rises almost 50 percent above the level it would be with full-smoothing commitment, and
period 2 consumption falls to just 37 percent of what it would be.} In other words, and despite having a first-mover advantage, Zero-self
can do little other than to partially accommodate the consumption
pattern that One-self wants to impose.

The contrast between monopoly and competition can be explained using
the intuition of income and substitution effects. Under monopoly, the no-renegotiation constraint can be viewed as a rise in the \emph{marginal} "price" of future utility (from Zero-self's perspective).\footnote{This intuition applies to CRRA and widely across other standard utility
functions, but there are exceptions. As shown in \citet{basu2020},
it is possible to construct utility functions where even as the overall
price of delivering future utility goes up, at some points the \emph{marginal}
price does not. In such cases, even under monopoly the change in contract
terms from full-smoothing to imperfect-smoothing commitment contract
could be ambiguous.} As a result, substitution effects will lead to an increase
in period 0 consumption and a drop in future consumption. Since the consumer is always left at her autarky utility, there are
no income effects. 

Under competition, on the other hand, the marginal price of future utility may rise or fall, so substitution effects are ambiguous (and tempered by opposing income effects). The resulting change in consumption patterns depends on parameter values and the shape of the consumer's utility function.

\subsection{Contracting with Naive Discounters}

For naive agents, the problem of renegotiation does not generally
lead to a renegotiation-proof contract. The naif believes she will
not be tempted to renegotiate.  Under monopoly, the
bank can add to its profits by engaging in renegotiation that was not
anticipated by the consumer in period 0. Under competition, banks
are led to return the potential surplus from renegotiation to the Zero-self.\footnote{A similar analysis could be carried out if consumers were misinformed
not about their own preferences but about $\kappa$.}

\subsubsection{Monopoly}

Relative to a sophisticated consumer, with a naive consumer the monopolist
bank can make additional profits on two margins. First, since there
is no perceived renegotiation problem, the consumer is willing to
accept a contract that is more profitable for the bank up-front; subsequently,
possible renegotiation generates additional profits for the bank.\footnote{There is an additional consideration \textendash{} that naive hyperbolic
discounters might be inaccurately optimistic about autarky outcomes
because of a failure to anticipate commitment problems. This would
have the interesting effect of tightening the participation constraint
and reducing surplus available to the monopolist.}

The bank must choose between a renegotiation-proof contract and one
that will be renegotiated upon. If $\kappa$ is sufficiently large
there is little to gain from renegotiation and the consumer will be
offered the full-smoothing commitment contract. But when $\kappa$
is relatively small, the bank might prefer to offer a contract that
will subsequently be renegotiated. In such cases, the bank solves
the following problem:
\begin{eqnarray}
\underset{C_{0}}{\text{max}} & \ \Pi_{0}\left(C_{0};Y_{0}\right)+\Pi_{1}\left(C_{1}^{m1}\left(C_{1}\right);C_{1}\right)-\kappa\\
s.t. & \ U_{0}\left(C_{0}\right)\geq U_{0}^{A}\label{eq:pc-n}
\end{eqnarray}

The solution, denoted $C_{0}^{mN}$, is explicitly derived
in the appendix (\ref{eq:naive-monopolist-contract1}, \ref{eq:naive-monopolist-contract2}).
The bank maximizes profits by offering a contract that divides future
consumption as much in favor of period 2 as possible. The greater
the imbalance between the contracted $c_{1}$ and $c_{2}$, the greater
the bank's profits from renegotiation. We show that if $\rho<1$,
the contract is at a corner solution where $c_{1}=0$. If $\rho>1$,
an explicit solution does not exist, but maximization pushes the contract
to a point where $c_{2}$ approaches infinity.\footnote{This assumes the consumer remains naive even in the face of such an incredible contract. While it is of interest to understand this benchmark case, a more realistic setup would incorporate limits on contract terms or on consumer credulity.} This contract can be compared to both the full-smoothing commitment
contract and the renegotiation-proof contract for sophisticates. In
particular, it will involve lower period 0 consumption than under
both full-smoothing commitment and renegotiation-proofness. This result
might appear counter-intuitive. In the case of lending, it does not
reinforce the narrative of banks preying on naive consumers by offering
them relatively large loans with steep repayments. Indeed, there are
other considerations beyond the scope of this model, such as the possibility
of collateral seizure, that could generate large loans. But our limited
model helps to highlight a particular aspect of contracting with naive
hyperbolic discounters: here, the bank offers them relatively \emph{small}
loans because its gains from renegotiation depend on the surplus that
the initial contract delivers to periods 1 and 2. In order to fully
take advantage of the consumer's naivete, the consumer must start
out with sufficiently small repayments that the bank could profit
from rearranging them.

The next proposition summarizes the above discussion. 
\begin{prop}
Suppose the consumer is naive. Under monopoly:
\begin{enumerate} [label=\alph*)]
\item If $\kappa$ is sufficiently higher than $\bar{\kappa}^{m}$,
the firm will offer the agent the full-smoothing commitment contract
($C_{0}^{mF}$) and it will not be renegotiated.

\item Otherwise, the contract $C_{0}^{mN}$ will satisfy $c_{0}^{mN}<c_{0}^{mF}<c_{0}^{mP}$
(either explicitly or in the limit), and it will be renegotiated in
period 1. 
\end{enumerate}
\end{prop}

\subsubsection{Competition}

Under competition with naive consumers, contracts must account for
renegotiation to have firms continue earning zero profits. Note that if contracts are not exclusive, the consumer gets offered
the full-smoothing commitment contract, which then gets renegotiated
if $\kappa<\bar{\kappa}$. This is because the firm offering the contract
in period 0 does not expect to benefit from renegotiation, so the
contract gets competed down to the one that maximizes the naive Zero-self's
perceived utility while delivering zero profits to the bank.

Under exclusive contracts, period 0 competition will imply that anticipated profits from future renegotiation
will be returned to the consumer through more favorable initial contracts.
If $\kappa$ is sufficiently small, the equilibrium contract involves
renegotiation and satisfies: 
\begin{align}
\underset{C_{0}}{\mbox{\ensuremath{max}}} & \ U_{0}\left(C_{0}\right)\\
s.t. & \ \Pi_{0}\left(C_{0};Y_{0}\right)+\Pi_{1}\left(C_{1}^{m1}\left(C_{1}\right);C_{1}\right)\geq\kappa
\end{align}

Let the solution be denoted $C_{0}^{N}$as explicitly derived
in the appendix (\ref{eq:naive-comp-contract1},  \ref{eq:naive-comp-contract2}).
As under monopoly, contracts divide future consumption as much in
favor of period 2 as possible to maximize the potential gains
from renegotiation. In the context of loans, this suggests contracts
where the debt burden is heaviest in the intermediate stages, resulting
in renegotiation to postpone payments.

Unlike under monopoly, competition for consumers returns anticipated renegotiation
gains to the consumer. Some of these gains are returned to the Zero-self,
so there is no clear prediction about whether period 0 consumption
will be lower or higher than under full-smoothing commitment. 
\begin{prop}
Suppose the consumer is naive. Under competition:
\begin{enumerate} [label=\alph*)]
\item If contracts are not exclusive: The consumer will accept the full-smoothing
commitment contract, $C_{0}^{F}$. The contract will be renegotiated
in period 1 if and only if $\kappa<\bar{\kappa}$.

\item If contracts are exclusive: 

\begin{enumerate} [label=\roman*)]
\item If $\kappa$ is sufficiently higher than $\bar{\kappa}$, the
consumer will accept the full-smoothing commitment contract ($C_{0}^{F}$)
and it will not be renegotiated.

\item Otherwise, the consumer will accept a contract $C_{0}^{N}$ with
the following properties: if $\rho<1$, $c_{0}^{N}<c_{0}^{F}$. If
$\rho>1$, then there are parameter values under which $c_{0}^{N}>c_{0}^{F}$. \end{enumerate}
\end{enumerate}
\end{prop}


\section{Commercial Nonprofits and Hybrid Ownership Forms}

\label{nonprofits}

Consider now the case of a firm that, in a pre-contract stage, has
the possibility of choosing its ownership structure as a way to tie its own hands. The firm might incorporate
as a legal non-profit or, more broadly, choose a degree of "hybrid"
ownership, for example by retaining for-profit status but actively attracting
social investors to establish considerable equity stakes and managerial
control. Similar to \citet{hansmann1996a} and as discussed
in the introduction, we think of these choices as imposing  restrictions on the firm's
ability to distribute profits to shareholders and managers in ways that can temper incentives. 
\begin{defn*}
Given "raw profits" $\Pi_{0}$, a "nonprofit" firm retains "captured
profits" $f\left(\Pi_{0}\right)$, where $f\left(0\right)=0$, $f'\left(\Pi_{0}\right)\in\left(0,1\right)$,
and $f''\left(\Pi_{0}\right)\leq0.$ 
\end{defn*}
This formulation is in the spirit of \citet{glaeser2001} who argued that although
the principals of a non-profit might technically be legally barred from
tying their own compensation to cash profits, in practice they often can capture a
fraction of those profits in imperfect and costly ways via the consumption
of perquisites or "dividends in kind" (e.g. the lavish expense account).
Our assumed shape of \(f\) captures the idea that the ability to employ perquisites and other such methods to substitute for unrestricted consumption
falls with profits. Alternatively, \(f\) could represent a firm's commitment to return some of its profits to its consumers or the larger community.

Setting aside mission or tax-advantage concerns that might additionally drive firms to adopt nonprofit
or hybrid status, we examine when purely profit-minded firms might
make such strategic governance choices; i.e. when will a voluntary restriction
on their ability to distribute profits make a self-interested firm better
off?\footnote{Indeed, welfare concerns could directly improve consumer outcomes
by either allowing Zero-self's participation constraint to be slack
or by raising the costs of renegotiation, $\kappa$. We consider the
latter point in an example below.} This has parallels to the explanation for commercial nonprofits due
to \citet{hansmann1996a} and modeled by \citet{glaeser2001} but
established on quite different behavioral grounds.\footnote{In those  accounts a firm delivers less than a promised quantity or
quality of a good or service, unambiguously harming the time-consistent
client. The client discovers this after the fact but cannot challenge
the contract breach only because it is too difficult or costly. In
contrast in our model the firm cannot unilateraly cheat the customer. Instead, the One-self customer both gain
from voluntarily breaking existing contract commitments and Zero-self
is no longer around to mount a challenge.}

At the outset, note that profit-oriented principals
have no incentive to switch to hybrid/nonprofit status when consumers
are naive. Since the consumer perceives no need for commitment,
any promise of superior commitment is of no value to her. 

With sophisticated consumers a firm, established as a commercial nonprofit, may have  an opportunity to extract greater surplus
from the consumer (by providing commitment), but now faces restrictions
on the ability to distribute this surplus to managers and shareholders.
This trade-off is sensitive to market structure. Under competition,
a lender's ability to provide effective commitment through non-profit
status depends on the exclusivity of contracts. When long-term contracts
can be made exclusive, the tradeoff disappears and all active firms
function as non-profits to attract customers. When contracts cannot be made exclusive, commitment generated through non-profit
status becomes impossible to achieve.

This might, for example, help explain a key difference between early microfinance
where  larger non-profit 
firms tended to dominate\ (e.g. Grameen Bank of Bangladesh, BRI Indonesia) offering rigid multi-period contracts, and say competitive
commercial credit card lending, or more competitive microfinance, which offer greater refinancing flexibility -- credit card punishments gain salience because they are\textit{ less}
(not more)\ strict, and therefore frequently triggered.

\subsection{Monopoly}

In a pre-contract phase the firm  establishes its type via
the adoption of legal nonprofit status and/or by choosing stable and credible
ownership and governance structures that commit it to profit distribution
limitations. If the monopolist were to operate as a nonprofit or
a hybrid (by adopting \( f\)), when facing a sophisticated hyperbolic discounter it would
design a renegotiation-proof contract to solve: 
\begin{align}
\underset{C_{0}}{\text{max}} & \ f\left(\Pi_{0}\left(C_{0};Y_{0}\right)\right)\\
\text{s.t.} & \ U_{0}\left(C_{0}\right)\geq U_{0}^{A}\\
 & \ f\left(\Pi\left(C_{0};Y_{0}\right)+\Pi_{1}\left(C_{1}^{m1}\left(C_{1}\right);C_{1}\right)\right)-f\left(\Pi\left(C_{0};Y_{0}\right)\right)\leq\kappa\label{eq:no-reneg-np}
\end{align}

Why operate as a nonprofit
when that reduces its ability to capture profits? The answer lies
in the loosening of the no-renegotiation constraint (\ref{eq:no-reneg-np}).
Now smaller gains from renegotiation are captured compared to the pure for-profit firm. Clearly, the pure for-profit monopolist's contract
($C_{0}^{mP})$ would leave the now more relaxed no-renegotiation constraint slack.
The nonprofit can offer to credibly commit to not renegotiate
contracts that offer greater consumption smoothing across periods
1 and 2, so Zero-self becomes more willing to pay for these smoother consumption
streams.

The "captured-profits" maximizing solution will be given by a contract that we denote $C_{0}^{mNP}$. If $\kappa<\bar{\kappa}^{m}$, with a relaxed
renegotiation-proof constraint $\Pi_{0}(C_{0}^{mNP};Y_{0})>\Pi_{0}(C_{0}^{mP};Y_{0})$
but whether or not it will be in the bank principals' best interest
to strategically convert to nonprofit status depends on whether the
captured profits under nonprofit status exceed the profits they could
earn as a pure for-profit; in other words on whether $f\left(\Pi_{0}(C_{0}^{mNP};Y_{0})\right)>f\left(\Pi_{0}(C_{0}^{mP};Y_{0})\right)$.
The monopolist faces a tradeoff in considering non-profit status:
higher raw profits (as the commitment problem is partly solved) but
a diminished capture of those raw profits.

Proposition 6 (in Section 6.2) establishes the
existence of captured profit functions that would be strictly preferred
to for-profit status for monopoly firms. 
We can also ask what types of
consumers are more likely to be served by such firms. If
consumers are far from optimal in autarky, then the pure for-profit firm
would make substantial profits. In such cases, the extra surplus from the nonprofit's
credibility advantage does not compensate for the  restrictions on profit distributions required.
 However, for consumers with higher autarky utility, we can describe situations where nonprofit or hybrid firms earn higher captured profts compared to pure for-profits. In some environments, nonprofits can operate where pure for-profits would fail to operate profitably at all.

\subsection{Competition}

\subsubsection{Exclusive contracts}

Consider the competitive market situation where contracts can be assumed to remain exclusive, so that any potential renegotiation surplus between the bank and One-self
goes to the bank.
In this setting, a nonprofit/hybrid firm will be led to offer contract
terms to solve: 

\begin{align}
\underset{C_{0}}{\text{max}} & \ U_{0}\left(C_{0}\right)\\
\text{s.t.} & \ f\left(\Pi_{0}(C_{0};Y_{0})\right)\geq0\\
 & \ f\left(\Pi\left(C_{0};Y_{0}\right)+\Pi_{1}\left(C_{1}^{m1}\left(C_{1}\right);C_{1}\right)\right)-f\left(\Pi\left(C_{0};Y_{0}\right)\right)\leq\kappa\label{eq:no-reneg-enp}
\end{align}

Label the contract that solves this program  $C_{0}^{eNP}$. Consider first a field where all firms start as pure for-profits
and earn zero profits. If the no-renegotiation constraint binds, Zero-self's
utility must be lower than optimal. Starting from this situation consider
now a firms' strategic choice to adopt nonprofit status.
Any firm that switches into nonprofit status can make positive
profits while offering Zero-self a contract with a higher discounted
utility because of the loosened no-renegotiation constraint (\ref{eq:no-reneg-enp}).
So, if the borrowers are sophisticated, in equilibrium,
all firms become nonprofit and earn zero profits.

\subsubsection{Non-Exclusive Contracts}

Now, assume that exclusivity, or  period 1 monopoly power, disappear.
Firms can compete to renegotiate each other's contracts in period
1.

If there were only nonprofits in equilibrium, any one firm could now make
positive profits by switching to for-profit status (or a new firm might enter), undoing a rival
bank's contract in period 1. As a result, equilibrium contracts will be
determined by for-profit firms, and consumers will be offered lower
commitment than from nonprofit firms alone.\footnote{The same argument applies if banks can costlessly renegotiate other bank's contracts.} 
The above discussion is summarized: 
\begin{prop}
Suppose the consumer is sophisticated.
\begin{enumerate} [label=\alph*)]
\item Suppose $\kappa<\bar{\kappa}^{m}$. Under monopoly, there exist
captured profit functions such that the firm will operate as a nonprofit.

\item Suppose $\kappa<\bar{\kappa}$. Under competition:
\begin{enumerate} [label=\roman*)]
\item If contracts
are exclusive, firms will operate as nonprofits for any captured profit
discount function.
\item If contracts are not exclusive, there is no
captured profit discount function under which firms will operate as
nonprofits. 
\end{enumerate}
\end{enumerate}
\end{prop}

\subsection{An example}

We close this section with an example that takes a broader view of the ways in which hybrid status might help a firm deliver commitment. Consider a firm that may choose its degree of
hybrid-ness or for-profit orientation, indexed by a parameter $\alpha\in\left[0,1\right]$.
For a chosen $\alpha$, captured profits are a linear function of raw profit:
\begin{equation}
f\left(\Pi_{0}\right)=\alpha\Pi_{0}
\end{equation}

An $\alpha=1$
would represent a pure for-profit investor-led firm, $\alpha=0$ a
strictly regulated non-profit.
We can also allow $\alpha$ to directly affect the non-pecuniary renegotiation
cost the firm's principals incur when they opportunistically break
contractual promises. A more hybrid or nonprofit firm
dominated by social investors is more likely to hire staff and managers
that internalize client welfare and social investor motivations and
are hence more likely to feel non-pecuniary costs associated with
guilt, cognitive dissonance, or loss of reputation from breaking promises. If we label the cost of renegotiation $\eta\left(\alpha\right)$ \textendash{}
replacing our earlier $\kappa$ \textendash{} this is captured
by assuming that function $\eta$ falls weakly in $\alpha$. Putting
both mechanisms together gives a modified no-renegotiation constraint: 
\begin{equation}
\alpha\Pi_{1}(C_{1}^{m1}(C_{1});C_{1})\leq\eta(\alpha)\label{eq:no_reneg_np}
\end{equation}
This states that the fraction of raw profits
$\Pi_{1}$ that can be captured from  contract renegotiation must
not exceed renegotiation costs. If we define $\kappa(\alpha)\equiv\frac{\eta(\alpha)}{\alpha}$, this
no-renegotiation constraint can be written as

\begin{equation}
\Pi_{1}(C_{1}^{m1}(C_{1});C_{1})\leq\kappa(\alpha)\label{eq:no-kalpha}
\end{equation}
which looks just like our earlier constraint (\ref{eq:rpc-m}) except $\kappa$
is now a function of $\alpha$. The earlier renegotiation problems
were for the special case of a pure for-profit firm with $\alpha=1$
but we can now analyze how contracting, captured profits, and client welfare
change with ownership/governance choices  $\alpha$  in a strategic equilibrium.

Note that if the loosening of the no-renegotiation constraint
happens primarily via the right-hand side (i.e. via term $\eta\left(\alpha\right)$,
which represents the firm's motivation to honor the initial agreement),
the firm benefits unambiguously as it is able to offer better
commitment \emph{and} fully retain the added profits.
This suggests that hiring caring, mission-oriented staff might be good for the firm's bottom line -- something that is not generally true in a market with only time-consistent clients.\footnote{Of course, the caring, mission-oriented staff are perhaps used towards cynical ends -- the firm offers better consumption smoothing but still extracts all surplus from the transaction.}
\begin{figure}
\includegraphics[scale=0.7]{Figure3.pdf}
\caption{\label{fig:nonprofit}Captured rent by ownership status and endowment
income}
\end{figure}

Figure \ref{fig:nonprofit} illustrates cases where non-pecuniary
costs to breaking promises fall with $\alpha$
according to $\eta(\alpha)=10 \cdot (1-\alpha)$ and hence $\kappa(\alpha)=10 \cdot \frac{(1-\alpha)}{\alpha}$.
The subplots show captured profits at different
levels of $\alpha$ for four  different consumer types, differentiated by their initial income streams.\footnote{Figure 3 parameters: $\beta=0.65$, $\rho=1.05$ and different income streams as indicated in the subplots.} In each case \(\beta=0.65\) and $\rho=1.05$ but consumers differ in terms of initial  income streams. All streams have present value of 300
but differ  in terms of period 0 income \(y_{0}\) with remaining income  divided equally between period 1 and 2, so \(y_1=y_2=\frac{300-y_0}{2}\). 
 The higher dotted curved lines in each subplot
represents `raw' profits $\Pi_{0}(C_{0}^{mNP};Y_{0})$ and the lower
solid curve represents captured profits $\alpha\Pi_{0}(C_{0}^{mNP};Y_{0})$. A horizontal
line has been drawn in to indicate the level of profits $\Pi_{0}(C_{0}^{mP};Y_{0})$
earned by a pure for-profit ($\alpha=1$ and $\kappa (1)=0$) and the shaded area indicates the extra profits captured by a hybrid firm of type \(\alpha\) compared to that pure-profit benchmark. Consider the top left panel
where the customer has initial income $(90,105,105)$. These  customers want to borrow significantly in period 0 so bank
profits would be large, even under the renegotiation-proof contract offered by the pure for-profit. Hybrid status, by lowering $\alpha$ and offering better commitment contracts, confers some profit gain.
For really heavy borrowers, with even lower \(y_{0}\) (not depicted), the gain to hybrid status become negligible  and  pure for-profits likely dominate. On the other hand, customers with endowments such as  $(140,80,80)$ are already fairly close
to their preferred consumption stream so the profits to be captured
even under full commitment are not that large. The pure and more for-profit firms cannot  earn positive profits but a broad range of nonprofits survive.  In a market where consumer demand for savings is higher (bottom right panel) we see  profits climb again and more commercially-oriented firms become more likely to prevail. 

\section{Discussion}

We discuss three areas of practical and policy concern that the
analysis aims to engage with.

\subsection{Commitment as a form of Consumer Protection}

Concerns about excessive refinancing and "over-indebtedness" have
been widely raised, especially in the lead up and wake of financial crises.
On the eve of the mortgage banking crisis of 2007, over 70 percent
of all new subprime mortgage loans were refinances of existing mortgages
and approximately 84 percent of these were "cash out" refinances \citep{demyanyk2011}.
In the market for payday loans in the United States economists and
regulatory observers express concern not so much that fees are high
(the typical cost is 15\% of the amount borrowed on a two week loan)
but that many payday loans are "rolled over" or renewed
rather than paid off at maturity, resulting in very high total loan costs and placing
many people into very difficult debt management situations \citep{deyoung2015}.

Consumer protection problems are often analyzed emphasizing two
broad channels: naive or uneducated consumers and their failure to correctly
anticipate fees and punishments (see \citet{gabaix2006},
\citet{armstrong2012}, and \citet{akerlof2015} for related arguments),
and/or bank's moral hazard (see \citet{dewatripont1999} and \citet{oak2010}).
We have argued that, given evidence of time-inconsistent
preferences,\footnote{See, for example, \citet{laibson2003}, \citet{ashraf2006},
\citet{gugerty2007}, and \citet{tanaka2010}.} a bank's ability to provide credible commitment should also fall
under this umbrella \textendash\  sometimes consumers \textit{want} punishments
or fees to limit renegotiation.

In light of consumer credit
market crises, there has been renewed emphasis on consumer protection and
better governance and regulation in banking.\footnote{In the US, the Consumer Financial Protection Bureau was set up in
2011 under the Dodd\textendash Frank Wall Street Reform and Consumer
Protection Act. In India, the far-reaching Micro Finance Institutions
Development and Regulation Bill of 2012 was designed to increase government
oversight of MFIs in response to the credit crisis in the state of
Andhra Pradesh, and the perception that lax consumer protection and
aggressive lending practices had led to rising over-indebtedness and
stress.} One particular outcome of concern has been borrower over-indebtedness,
an issue that has been at the center of recent microfinance repayment
crises in places as far-flung as Morocco, Bosnia, Nicaragua and India,
as well as the 2008 mortgage lending crisis in the United States.
In each of these cases the issue of refinancing or the taking of loans
from multiple lenders emerges.

Journalistic and scholarly analyses of such situations, including
the recent mortgage crisis in the United States, often frame
the issues as problems of consumer protection, implying that lenders design products to purposefully take advantage of borrowers
who have limited financial literacy skills and are naive about their
self-control problems. Informed by such interpretations, regulations
introduced in the wake of these crises swung toward restricting
the terms of allowable contracts, for example by setting maximum interest
rates and limiting the use of coercive loan recovery methods.

We too place consumers' struggles with intertemporal self-control issues
at the center of the analysis, but with an emphasis on \emph{sophisticated} borrowers. From this perspective, "predatory lending"
is not primarily about tricking naive borrowers into paying more than
they signed up for with hidden penalties or misleading interest rates
quotes, but about offering excessive flexibility and refinancing of
financial contracts in ways that may undermine the commitments
to long term consumption and debt management paths that borrowers
may want to put in place.\footnote{\citet{bond2009} discuss evidence of predatory lending in the context
of mortgages. In 2016 the Consumer Financial Protection Bureau put
forth a proposal to protect payday loan consumers including limits
on the number and frequency of re-borrowings \citep{cfpb2016}.}

Here, a bank that promises to be rigid and is then flexible could
be seen as hurting, rather than helping, the consumer. We take seriously
the bank's ex-post considerations and derive conditions under which
it would renegotiate.
In this sense, our paper complements some others that demonstrate
how commitment can be undone in related settings.\footnote{\citet{gottlieb2008}
shows how competition leads to inefficient outcomes in immediate rewards
goods. \citet{heidhues2010} study the mistakes of partially naive
borrowers in competitive credit markets.  \citet{mendez2012} analyzes
predatory lending with naive consumers.} This also leaves open for future research the question of how, with heterogenous populations, banks might balance their incentive to discourage renegotiation (for sophisticated hyperbolic discounters) and to encourage renegotiation (for naive hyperbolic discounters and other uninformed consumers).

\subsection{Commercial Non-profits and Hybrid Ownership in Finance}

Henry \citet{hansmann1996a} argued that in markets where the
quality of products or services was difficult to verify, clients
would rationally fear that investor-led firms might opportunistically
skimp on the quality of a promised product or service, or reveal a
hidden fee, and that this could greatly reduce or even eliminate contracting. "Commercial non-profit" status might then be a
costly but necessary strategy by firms to commit to not act opportunistically,
hence enabling trade.
Hansmann used as a primary example the development of
consumer saving, lending and insurance products in the United States
and Europe. Life insurance in the United States was, until
quite recently, dominated by mutuals. In the absence of government consumer protection, rate payers might
not trust investor-led firms to not act  opportunistically by, for
example, by skimping or reneging on insurance
payouts. Mutuals  had less incentive to  increase shareholder dividends this way as the clients themselves are the
only shareholders. Mutuals therefore enjoyed a distinct competitive
advantage until sufficient state regulatory capacity developed.

We follow Hansmann in thinking of a nonprofit as "in
essence, an organization that is barred from distributing its net
earnings, if any, to individuals who exercise control over it, such
as members, officers, directors, or trustees."\footnote{Hence we abstract away from other considerations for nonprofits, as
in \citet{besley2005}, \citet{mcintosh2005}, and \citet{guha2013}.
Nonetheless our modeling framework could be adapted to include these
considerations. Nonprofit firms also often enjoy tax and other benefits denied to for-profit firms (see, for example, Cohen, 2015). But for
our argument, it is the \textit{restrictions},
not benefits, that generate improved outcomes.} \citet{glaeser2001} formalized Hansmann's central argument
to show that when a firm cannot commit to maintaining high quality,
it might choose to operate as a commercial nonprofit rather than as
an investor-led for-profit to more credibly signal that it has
weaker incentives to cheat on unobserved product
quality. As Hansmann explains, firm ownership form adapts endogenously
to serve as a ``crude form of consumer protection'' in unregulated emerging
markets where asymmetric information problems are rife. \citet{bubb2013}
modify this model to include behavioral borrowers so that the non-contractible quality issue is on
hidden penalties, which are incurred with certainty by some borrowers.
All of these models are built to rely on some form of asymmetric information
or contract verification problem.

This paper argues that a theory of ownership
form can be built on behavioral micro-foundations even in environments
with no asymmetric information and with sophisticated forward-looking
agents. We extended the  argument to  include  hybrid ownership forms that still clearly dominate the sector in most developing countries
\citep{cull2009,conning2011}. Hybrid ownership is common in microfinance where, for example, a lender might be legally incorporated as a for-profit and attract equity investors, but in practice their board is, by  design, dominated by social investors or client representatives  who  exert substantial
control and emphasize a "double bottom line."\footnote{Social investors include international financial institutions and mission-oriented or ethical investment funds such as OikoCredit,  Calvert Funds, etc.}
Hybrid ownership  thus appears to confer
many of the benefits of nonprofit status -- specifically, credible
commitment to consumer protection -- with fewer of the costs. In particular,
unlike a pure nonprofit, hybrid firms can and do have  outside equity investors, but the firms' relative emphasis on shareholder value versus client welfare can be adjusted in part via the relative mix of commercial and social investors and the resulting effect on firm governance. 
\subsection{Market Structure and Governance Choice}



Commenting upon a major microfinance crisis in the state of Andhra
Pradesh in India, veteran microfinance market investor and analyst
Elizabeth \citet{rhyne2011} describes the build up of ``rising debt
stress among possibly tens of thousands of clients, brought on by
explosive growth of microfinance organizations . . .\textquotedblright{}
fueled by the rapid inflow of directed private lending and new equity
investors who, because they ``paid dearly for shares in {[}newly
privatized{]} MFIs . . . needed fast growth to make their investments
pay off.\textquotedblright{}

She goes on to lay the blame on ``poor governance frameworks\textquotedblright{}
for behaviors that included ``loan officers {[}that{]} often sell
loans to clients already indebted to other organizations.'' In her
view, Indian MFIs might have avoided their problems and followed the
model of leading microfinance organizations in other countries like
Mibanco (Peru) and Bancosol (Bolivia) which ``were commercialized
with a mix of owners including the original non-governmental organization
(NGO), international social investors (including development banks),
and some local shareholders. The NGOs kept the focus on the mission,
while the international social investors contributed a commercial
orientation, also tempered by social mission.'' These are the types
of hybrid ownership forms, along with nonprofit firms, that we argue
can provide surplus-building consumer protection through a reduced
incentive to renegotiate. Rhyne's argument is that a number of Indian
state regulations made it difficult for such hybrid ownership forms
to rise organically in India. Our model introduces an additional consideration -- these governance
choices are highly dependent on market structure, and nonprofit or hybrid firms may not survive in markets where banks and MFIs proliferate and contracts are non-exclusive, as experienced in many economically developing regions within and outside India.

\section{Conclusion}

The starting point for this paper is the observation that the solution
to any commitment problem must also address a renegotiation problem.
We show how the renegotiation problem depends on costs of renegotiation
and how it changes contract terms in sometimes unexpected ways. In
this context, we also provide a rationalization of commercial nonprofits
in the absence of asymmetric information.

We argue that the model sheds some light on trends in microfinance,
payday lending, and mortgage lending. We hope this paper also offers
a framework that can be built upon. The incorporation of additional
"real-world" factors could improve our understanding of particular
institutions and generate empirically relevant comparative statics.
Examples of these include nondeterministic incomes, private and heterogeneous types, collateral and strategic default, and longer time horizons.

In this paper, we make the simplifying assumption that new contracts
can only be signed in period 0. This assumption is inconsequential
under competition but has some relevance under monopoly. As a result of the assumption,
in the profit maximization problem the consumer\textquoteright s outside
option is the same as her autarky consumption. This streamlines the
analysis but could easily be lifted without altering the intuition
of the model. If fresh contracts could be signed in future periods,
the consumer\textquoteright s participation constraint would have
to take these into account. \citet{basu2020} shows that the possibility
of future contracts can affect current participation constraints in
some subtle ways\textemdash the outside option is not monotonic in
autarky utility and could be strictly better or strictly worse than
autarky. However, for the purposes of this paper, \emph{given} an outside
option, even if its relationship to autarky is complex, the optimization
problems remain as specified.

Finally, the differences between monopoly and competition open up
some new, potentially interesting questions. How does market structure
evolve and what are the implications for commitment? And through this
evolution might there emerge third parties to contracts between consumers
and banks that can more effectively enforce the commitment that is
sought after on both sides of the market?

\appendix

\section{Appendix: CRRA Derivations and Proofs}

\subsection{Full-commitment}

\subsubsection{Competition}

Combining the first-order conditions (\ref{eq:FOC_comp}) and the
budget constraint (\ref{eq:BPC0}) of the utility maximization problem,
the competitive full-smoothing commitment contract $C_{0}^{F}$ is:
\begin{equation}
C_{0}^{F}=\left(\frac{y}{1+2\beta^{\frac{1}{\rho}}}\right)\cdot\left(1,\beta^{\frac{1}{\rho}},\beta^{\frac{1}{\rho}}\right)\label{eq:c-f}
\end{equation}


\subsubsection{Monopoly}

For the monopolist bank that offers full-commitment, the solution
is determined by the first-order condition and the consumer's participation
constraint: 
\begin{align}
C_{0}^{mF} & =\left(\frac{U_{0}^{A}\left(1-\rho\right)}{1+2\beta^{\frac{1}{\rho}}}\right)^{\frac{1}{1-\rho}}\cdot\left(1,\beta^{\frac{1}{\rho}},\beta^{\frac{1}{\rho}}\right)\label{eq:c-mf}\\
\Pi_{0}\left(C_{0}^{mF};Y_{0}\right) & =y-\left(U_{0}^{A}\left(1-\rho\right)\right)^{\frac{1}{1-\rho}}\left(1+2\beta^{\frac{1}{\rho}}\right)^{\frac{-\rho}{1-\rho}}\label{eq:pi-mf}
\end{align}

It can easily be verified that $C_{0}^{F}>C_{0}^{mF}$.

\subsection{The no-renegotiation constraint }

Consider any existing continuation contract $C_{1}^{0}$. The competitively
renegotiated contract (most beneficial to the consumer) will be: 
\begin{equation}
C_{1}^{1}\left(C_{1}^{0}\right)=\left(\frac{c_{1}^{0}+c_{2}^{0}-\kappa}{1+\beta^{\frac{1}{\rho}}}\right)\cdot\left(1,\beta^{\frac{1}{\rho}}\right)\label{eq:c-r}
\end{equation}
The condition to make sure the consumer will neither propose nor accept
this most favorable renegotiation is:

\begin{equation}
U(C_{1}^{1}\left(C_{1}^{0}\right))\le U(C_{1}^{0})\label{eq:uc-r}
\end{equation}
Substituting (\ref{eq:c-r}) into this and re-arranging allows us to
write the no-renegotiation constraint as the condition:
\begin{equation}
u(c_{1}^{0})+\beta u(c_{2}^{0})\ge(1+\beta^{\frac{1}{\rho}})u\left(\frac{c_{1}^{0}+c_{2}^{0}-\kappa}{1+\beta^{\frac{1}{\rho}}}\right)\label{eq:no-reg-equal}
\end{equation}
The same no-renegotiation constraint can be derived starting from
the assumption of period 1 monopoly. The most favorable renegotiation
for the monopolist is:
\begin{equation}
C_{1}^{m1}\left(C_{1}^{0}\right)=\left(\frac{(c_{1}^{0})^{1-\rho}+\beta(c_{2}^{0})^{1-\rho}}{1+\beta^{\frac{1}{\rho}}}\right)^{\frac{1}{1-\rho}}\cdot\left(1,\beta^{\frac{1}{\rho}}\right)\label{eq:m-r}
\end{equation}
The contract will not be renegotiated so long as the profits gains
to this most favorable renegotiation fall short of renegotiation costs:
\begin{equation}
\Pi_{1}\left(C_{1}^{m1}\left(C_{1}^{0}\right);C_{1}^{0}\right)=\left(c_{1}^{0}+c_{2}^{0}-\kappa\right)-\left((c_{1}^{0})^{1-\rho}+\beta(c_{2}^{0})^{1-\rho}\right)^{\frac{1}{1-\rho}}\left(1+\beta^{\frac{1}{\rho}}\right)^{\frac{-\rho}{1-\rho}}\le\kappa\label{eq:pi-r}
\end{equation}
This can be rearranged to yield the same condition as (\ref{eq:no-reg-equal}).

\subsubsection{No-renegotiation condition}

Substituting from (\ref{eq:pi-r}) in the no-renegotiation condition
(\ref{eq:no-reg-monop}), we get the following explicit no-renegotiation
condition: 
\begin{equation}
u(c_{1}^{0})+\beta u(c_{2}^{0})\le u\left(\frac{c_{1}^{0}+c_{2}^{0}-\kappa}{1+\beta^{\frac{1}{\rho}}}\right)(1+\beta^{\frac{1}{\rho}})\label{eq:no-renegotiation}
\end{equation}
This condition applies identically whether contract renegotiation
happens under competition or monopoly.

\subsubsection{No-renegotiation condition for full-smoothing contracts}

Setting $c_{1}^{0}=c_{2}^{0}$ in the no-renegotiation constraint
(\ref{eq:no-renegotiation}) above we can re-arrange the constraint
as: 

\begin{equation}
\kappa\geq c_{1}^{0}\cdotp\Upsilon\label{eq:smoothing-no-renegotiation}
\end{equation}

where 
\begin{equation}
\Upsilon=\left[2-\left[\frac{(1+\beta)}{\left(1+\beta^{\frac{1}{\rho}}\right)^{\rho}}\right]^{\frac{1}{1-\rho}}\right]\label{eq:upsilon}
\end{equation}


\subsection{Imperfect-Smoothing Commitment Contracts}

Redefine any consumption stream in the following manner: 
\begin{equation}
C_{0}=\left(c_{0},c_{1},c_{2}\right)\equiv\left(c_{0},\alpha s,\left(1-\alpha\right)s\right)\label{eq:new-notation}
\end{equation}
so that $c_{1}$ and $c_{2}$ are expressed as shares of total future
consumption $s$. Since the no-renegotiation constraint places restrictions
on the relative values of $c_{1}$ and $c_{2}$, we can rewrite the
constraint (\ref{eq:no-renegotiation}) using the new notation to get
a continuous function $\alpha\left(s\right)$, which determines the
minimum fraction of any $s$ that must be offered to One-self to prevent
renegotiation: 
\begin{equation}
\left(s\right)\left(1-\left(\alpha^{1-\rho}+\beta\left(1-\alpha\right)^{1-\rho}\right)^{\frac{1}{1-\rho}}\left(1+\beta^{\frac{1}{\rho}}\right)^{\frac{-\rho}{1-\rho}}\right)\leq\kappa\label{eq:no-renegotiation-alpha-s}
\end{equation}

Observe that at One-self's optimal division of $s$, $\left(c_{2}=\beta^{\frac{1}{\rho}}c_{1}\Longleftrightarrow\alpha=\frac{1}{1+\beta^{\frac{1}{\rho}}}\right)$
, there cannot be profit gains from renegotiation so the constraint
will be slack. For any $s$, there may be two values of $\alpha$
that satisfy the constraint with equality\textendash one with $\alpha$
smaller than One-self would like (lower boundary), and another with
$\alpha$ larger than One-self would like (upper boundary). Assuming
the full-smoothing contract does not satisfy the constraint, the second-best
contract must lie on the lower boundary. This defines a continuous
function $\alpha\left(s\right)$, which determines the minimum fraction
of any $s$ that must be offered to One-self to prevent renegotiation.
\begin{equation}
\alpha\left(s\right)=min\left\{ \alpha:\left(s\right)\left(1-\left(\alpha^{1-\rho}+\beta\left(1-\alpha\right)^{1-\rho}\right)^{\frac{1}{1-\rho}}\left(1+\beta^{\frac{1}{\rho}}\right)^{\frac{-\rho}{1-\rho}}\right)=\kappa\right\} \label{eq:alpha-1}
\end{equation}

It can easily be verified that $\alpha'\left(s\right)>0$ (profits
from renegotiation rise in $s$, so if $s$ rises there must be an
increase in the share allocated to One-self to compensate). Implicitly
differentiating the binding no-renegotiation constraint by $s$, we
have: 
\begin{equation}
\frac{d\alpha}{ds}=\left(\frac{k}{s^{2}}\right)\left(\frac{1+\beta^{\frac{1}{\rho}}}{\alpha^{1-\rho}+\beta\left(1-\alpha\right)^{1-\rho}}\right)^{\frac{\rho}{1-\rho}}\left(\frac{1}{\alpha^{-\rho}-\beta\left(1-\alpha\right)^{-\rho}}\right)\label{eq:dalpha-ds}
\end{equation}

The terms in the first two sets of parentheses are always positive.
The last term is positive when the no-renegotiation constraint is
binding (One-self would ideally like $\alpha^{-\rho}=\beta\left(1-\alpha\right)^{-\rho}$
but if $\kappa>0$ she has to settle for $\alpha^{-\rho}>\beta\left(1-\alpha\right)^{-\rho}$.

Finally, for any $s$ and $\alpha$, let 
\begin{equation}
V\left(s,\alpha\right)\equiv\beta\left[u\left(\alpha s\right)+u\left(\left(1-\alpha\right)s\right)\right]\label{eq:cont-utility}
\end{equation}
This is the discounted utility over periods 1 and 2, from period 0's
perspective. It will be useful to note that the first-order conditions
of the full-smoothing contract problems (competition and monopoly)
can be written as: 
\begin{equation}
\frac{du\left(c_{0}\right)}{dc_{0}}=\frac{dV\left(s,\frac{1}{2}\right)}{ds}\label{eq:FOC-with-V}
\end{equation}


\subsubsection{Sophisticated Hyperbolic Discounters}

\textit{Proof of Proposition 2:}
\begin{enumerate} [label=\alph*)]
\item Since the full-commitment profit-maximizing
contract was uniquely determined, and since it does not satisfy the
no-renegotiation constraint, the renegotiation-proof contract
must yield lower profits than the full-commitment contract does.

\item Using the modified notation, the full-smoothing contract terms
are $c_{0}^{mF}$ and $s^{mF}$, with $\alpha^{mF}=\frac{1}{2}$.
The imperfect-smoothing contract terms are $c_{0}^{mP}$ and $s^{mP}$,
with $\alpha^{mP}=\alpha\left(s^{mP}\right)$. Suppose $c_{0}^{mP}\leq c_{0}^{mF}$.
Then, to satisfy Zero-self's participation constraint, 
\begin{align}
V\left(s^{mP},\alpha\left(s^{mP}\right)\right) & \geq V\left(s^{mF},\frac{1}{2}\right)\\
\Rightarrow s^{mP} & \geq s^{mF}\left[\frac{\left(\frac{1}{2}\right)^{1-\rho}+\left(\frac{1}{2}\right)^{1-\rho}}{\left(\alpha^{mP}\right)^{1-\rho}+\left(1-\alpha^{mP}\right)^{1-\rho}}\right]^{\frac{1}{1-\rho}}\label{eq:s-compare}
\end{align}

Differentiating $V\left(s^{mP},\alpha^{mP}\right)$, we get the following
inequalities:\footnote{An explanation of the steps: (\ref{eq:deriv2}) follows from the
fact that $\alpha\left(s\right)$ rises in $s$ (derived from Equation
\ref{eq:alpha-1}) and $V$ falls as $\alpha$ rises, making the allocation
worse from Zero-self's perspective. (\ref{eq:deriv4}) follows
from (\ref{eq:s-compare}). (\ref{eq:deriv8}) follows
from the FOC of the monopolist's profit-maximization problem with
full-smoothing contracts. } 
\begin{align}
\frac{dV\left(s^{mP},\alpha^{mP}\right)}{ds} & =\frac{\partial V\left(s^{mP},\alpha^{mP}\right)}{\partial s}+\frac{\partial V\left(s^{mP},\alpha^{mP}\right)}{\partial\alpha}\frac{d\alpha^{mP}}{ds}\label{eq:deriv1}\\
 & <\frac{\partial V\left(s^{mP},\alpha^{mP}\right)}{\partial s}\label{eq:deriv2}\\
 & =\beta\left(s^{mP}\right)^{-\rho}\left[\left(\alpha^{mP}\right)^{1-\rho}+\left(1-\alpha^{mP}\right)^{1-\rho}\right]\label{eq:deriv3}\\
 & \leq\beta\left(s^{mF}\right)^{-\rho}\left[\frac{\left(\frac{1}{2}\right)^{1-\rho}+\left(\frac{1}{2}\right)^{1-\rho}}{\left(\alpha^{mP}\right)^{1-\rho}+\left(1-\alpha^{mP}\right)^{1-\rho}}\right]^{\frac{-\rho}{1-\rho}}\left[\left(\alpha^{mP}\right)^{1-\rho}+\left(1-\alpha^{mP}\right)^{1-\rho}\right]\label{eq:deriv4}\\
 & =\beta\left(s^{mF}\right)^{-\rho}\left[\left(\frac{1}{2}\right)^{1-\rho}+\left(\frac{1}{2}\right)^{1-\rho}\right]\left[\frac{\left(\alpha^{mP}\right)^{1-\rho}+\left(1-\alpha^{mP}\right)^{1-\rho}}{\left(\frac{1}{2}\right)^{1-\rho}+\left(\frac{1}{2}\right)^{1-\rho}}\right]^{\frac{1}{1-\rho}}\label{eq:deriv5}\\
 & <\beta\left(s^{mF}\right)^{-\rho}\left[\left(\frac{1}{2}\right)^{1-\rho}+\left(\frac{1}{2}\right)^{1-\rho}\right]\label{eq:deriv6}\\
 & =\frac{dV\left(s^{mF},\alpha^{mF}\right)}{ds}\label{eq:deriv7}\\
 & =\frac{du\left(c_{0}^{mF}\right)}{dc_{0}^{mF}}\label{eq:deriv8}\\
 & \leq\frac{du\left(c_{0}^{mP}\right)}{dc_{0}^{mP}}\label{eq:deriv9} 
\end{align}

Since $\frac{dV\left(s^{mP},\alpha^{mP}\right)}{ds}<\frac{du\left(c_{0}^{mP}\right)}{dc_{0}^{mP}}$,
this contract cannot be profit maximizing for the monopolist (it could
do better by reallocating consumption away towards Zero-self). This
contradiction implies that our assumption is incorrect. It must be
true that at the profit-maximizing imperfect-smoothing contract, $c_{0}^{mP}>c_{0}^{mF}$.
$\Square$
\end{enumerate}

\emph{Proof of Proposition 3:}
\begin{enumerate} [label=\alph*)]
\item We know that $U_{0}\left(C_{0}^{F}\right)=U_{0}^{F}$.
By assumption, since the no-renegotiation constraint is binding,
the renegotiation-proof contract cannot offer the optimal consumption
path. Therefore $U_{0}\left(C_{0}^{P}\right)<U_{0}\left(C_{0}^{F}\right)$.

\item At the full-commitment contract: 
\begin{equation}
\frac{du\left(c_{0}^{F}\right)}{dc_{0}}=\frac{dV\left(s^{F},\frac{1}{2}\right)}{ds}=\left(s^{F}\right)^{-\rho}\left(2\left(\frac{1}{2}\right)^{1-\rho}\right)
\end{equation}
Consider a renegotiation-proof contract with $c_{0}=c_{0}^{F}$. To
keep bank profits zero, this contract would also have $s=s^{F}$.
But in the renegotiation-proof contract, $s$ must be divided according
to $\alpha\left(s^{F}\right)$. So: 
\begin{align}
\frac{dV\left(s^{F},\alpha\left(s^{F}\right)\right)}{ds} & =\left(s^{F}\right)^{-\rho}\left(\alpha\left(s^{F}\right)^{1-\rho}+\left(1-\alpha\left(s^{F}\right)\right)^{1-\rho}\right)\nonumber \\
 & +\frac{d\alpha\left(s^{F}\right)}{ds}\left(s^{F}\right)^{1-\rho}\left(\alpha\left(s^{F}\right)^{-\rho}-\left(1-\alpha\left(s^{F}\right)\right)^{-\rho}\right)\label{eq:dv-ds-comp}
\end{align}
The first term -- the direct effect of a change in $s$ -- is
weakly less than $\frac{dV\left(s^{F},\frac{1}{2}\right)}{ds}$ if
$\rho\leq1$ and strictly greater if $\rho>1$. The second term -- the
component of $\frac{dV}{ds}$ that is driven by the change in $\alpha$ -- is
strictly negative. Therefore, if $\rho<1$, $\frac{dV\left(s^{F},\alpha\left(s^{F}\right)\right)}{ds}<\frac{dV\left(s^{F},\frac{1}{2}\right)}{ds}=\frac{du\left(c_{0}^{F}\right)}{dc}$,
so the renegotiation-proof contract must satisfy $c_{0}^{P}>c_{0}^{F}$.

Next, we consider the case when $\rho>1$. We can make the following
observations about $\alpha\left(s\right)$. First, $\underset{\kappa\rightarrow0}{lim}\alpha\left(s\right)=\frac{\beta^{\frac{-1}{\rho}}}{1+\beta^{\frac{-1}{\rho}}}$.
Second, implicitly differentiating  (\ref{eq:no-renegotiation-alpha-s})
with respect to $s$, and combining it with the previous limit result,
we get $\underset{\kappa\rightarrow0}{lim}\frac{d\alpha\left(s\right)}{ds}=0$.
Therefore, if $\rho>1$ and $\kappa$ is small enough, the second
term in (\ref{eq:dv-ds-comp}) will be sufficiently small in
magnitude that $\frac{dV\left(s^{F},\alpha\left(s^{F}\right)\right)}{ds}>\frac{dV\left(s^{F},\frac{1}{2}\right)}{ds}=\frac{du\left(c_{0}^{F}\right)}{dc}$.
In this case, the renegotiation-proof contract must satisfy $c_{0}^{P}<c_{0}^{F}$.
$\Square$
\end{enumerate}

If $\kappa=0$, the renegotiation-proof contracts can be explicitly
derived since in any contract it must be true that $c_{2}=\beta^{\frac{1}{\rho}}c_{1}$.
Solving the respective maximization problems, we get the following
equilibrium contracts for monopoly and competition, respectively:
\begin{align}
C_{0}^{mP}= & \left(\left(\frac{U_{0}^{A}\left(1-\rho\right)}{1+\beta^{\frac{1}{\rho}}\left(\frac{\left(1+\beta^{\frac{1-\rho}{\rho}}\right)^{\frac{1}{\rho}}}{\left(1+\beta^{\frac{1}{\rho}}\right)^{\frac{1-\rho}{\rho}}}\right)}\right)^{\frac{1}{1-\rho}},\left(\frac{\beta+\beta^{\frac{1}{\rho}}}{1+\beta^{\frac{1}{\rho}}}\right)^{\frac{1}{\rho}}c_{0}^{mP},\beta^{\frac{1}{\rho}}\left(\frac{\beta+\beta^{\frac{1}{\rho}}}{1+\beta^{\frac{1}{\rho}}}\right)^{\frac{1}{\rho}}c_{0}^{mP}\right)\label{eq:zerokappa-monop}\\
C_{0}^{P}= & \left(\frac{y}{1+\beta+\beta^{\frac{1}{\rho}}},\left(\frac{\beta+\beta^{\frac{1}{\rho}}}{1+\beta^{\frac{1}{\rho}}}\right)c_{0}^{P},\beta^{\frac{1}{\rho}}\left(\frac{\beta+\beta^{\frac{1}{\rho}}}{1+\beta^{\frac{1}{\rho}}}\right)c_{0}^{P}\right)\label{eq:zerokappa-comp}
\end{align}

It can easily be established that $c_{0}^{mP}>c_{0}^{mF}$, $c_{0}^{P}>c_{0}^{mF}$
if $\rho>1$, and $c_{0}^{P}<c_{0}^{mF}$ if $\rho<1$.

\subsubsection{Naive Hyperbolic Discounters}

Suppose the monopolist intends to renegotiate the contract. The maximization
problem, combined with the expression for $C_{1}^{m1}\left(C_{1}\right)$
(\ref{eq:m-r}), simplifies to:

\begin{align}
\underset{c_{0},c_{1},c_{2}}{max} & y-c_{0}-\frac{\left(c_{1}^{1-\rho}+\beta c_{2}^{1-\rho}\right)^{\frac{1}{1-\rho}}}{\left(1+\beta^{\frac{1}{\rho}}\right)^{\frac{\rho}{1-\rho}}}-\kappa\\
s.t. & \frac{c_{0}^{1-\rho}}{1-\rho}+\beta\frac{c_{1}^{1-\rho}}{1-\rho}+\beta\frac{c_{2}^{1-\rho}}{1-\rho}\geq U_{0}^{A}
\end{align}

The partial derivatives of the resulting Lagrangian are: 
\begin{align}
\frac{\partial\mathcal{L}}{\partial c_{0}} & =-1-\lambda c_{0}^{-\rho}\label{eq:L0}\\
\frac{\partial\mathcal{L}}{\partial c_{1}} & =c_{1}^{-\rho}\left[-\left(\frac{c_{1}^{1-\rho}+\beta c_{2}^{1-\rho}}{1+\beta^{\frac{1}{\rho}}}\right)^{\frac{\rho}{1-\rho}}-\lambda\beta\right]\label{eq:L1}\\
\frac{\partial\mathcal{L}}{\partial c_{2}} & =c_{2}^{-\rho}\left[-\beta\left(\frac{c_{1}^{1-\rho}+\beta c_{2}^{1-\rho}}{1+\beta^{\frac{1}{\rho}}}\right)^{\frac{\rho}{1-\rho}}-\lambda\beta\right]\label{eq:L2}
\end{align}

An interior solution, with $\frac{\partial\mathcal{L}}{\partial c_{1}}=0$
and $\frac{\partial\mathcal{L}}{\partial c_{2}}=0$ does not exist
(on a $c_{1}-c_{2}$ plot, the two first-order conditions do not intersect).
If $\rho<1$, the Lagrangian is maximized at a corner solution with
$c_{1}=0$. If $\rho>1$, the Lagrangian is maximized at the limit
as $c_{2}$ approaches infinity. Using this, the maximization problem
can be re-solved. If $\rho<1$: 
\begin{equation}
C_{0}^{mN}=\left(\left(\frac{U_{0}^{A}\left(1-\rho\right)}{2+\beta^{\frac{1}{\rho}}}\right)^{\frac{1}{1-\rho}},0,\left(\frac{1+\beta^{\frac{1}{\rho}}}{\beta}\right)^{\frac{1}{1-\rho}}\left(\frac{U_{0}^{A}\left(1-\rho\right)}{2+\beta^{\frac{1}{\rho}}}\right)^{\frac{1}{1-\rho}}\right)\label{eq:naive-monopolist-contract1}
\end{equation}
If $\rho>1$, the solution is undefined, but in the limit is given
by: 
\begin{equation}
C_{0}^{mN}=\left(\left(\frac{U_{0}^{A}\left(1-\rho\right)}{1+\left(1+\beta^{\frac{1}{\rho}}\right)\beta^{\frac{1}{\rho}}}\right)^{\frac{1}{1-\rho}},\beta^{\frac{1}{\rho}}\left(1+\beta^{\frac{1}{\rho}}\right)^{\frac{1}{1-\rho}}\left(\frac{U_{0}^{A}\left(1-\rho\right)}{1+\left(1+\beta^{\frac{1}{\rho}}\right)\beta^{\frac{1}{\rho}}}\right)^{\frac{1}{1-\rho}},\infty\right)\label{eq:naive-monopolist-contract2}
\end{equation}

Let us define profits from such a contract as: 
\[
\Pi_{0}^{mN}\equiv\Pi\left(C_{0}^{mN};Y_{0}\right)+\Pi_{1}\left(C_{1}^{m1}\left(C_{1}^{mN}\right);C_{1}^{mN}\right)-\kappa
\]

\emph{Proof of Proposition 4:} (a) and (b) are simultaneously established
through the following observations. First, $\Pi_{0}^{mN}$ is strictly
falling in $\kappa$ while $\Pi_{0}\left(C_{0}^{mF};Y_{0}\right)$
is invariant in $\kappa$. Second, at $\kappa=\bar{\kappa}^{m}$,
\begin{equation}
\Pi_{0}\left(C_{0}^{mF};Y_{0}\right)=\Pi_{0}\left(C_{0}^{mF};Y_{0}\right)+\Pi_{1}\left(C_{1}^{m1}\left(C_{1}^{mF}\right);C_{1}^{mF}\right)-\kappa<\Pi_{0}^{mN}
\end{equation}
Third, if $\kappa$ gets indefinitely large, $\Pi_{0}\left(C_{0}^{mF};Y_{0}\right)>\Pi_{0}^{mN}$.
Finally, it can be verified from the explicit derivations that $c_{0}^{mN}<c_{0}^{mF}$.
$\Square$

We now derive equilibrium contracts for naive consumers under perfect
competition. Suppose contracts are exclusive. Then, a contract that
is renegotiated satisfies: 
\begin{align}
\underset{c_{0},c_{1},c_{2}}{max} & \frac{c_{0}^{1-\rho}}{1-\rho}+\beta\frac{c_{1}^{1-\rho}}{1-\rho}+\beta\frac{c_{2}^{1-\rho}}{1-\rho}\\
s.t. & y-c_{0}-\frac{\left(c_{1}^{1-\rho}+\beta c_{2}^{1-\rho}\right)^{\frac{1}{1-\rho}}}{\left(1+\beta^{\frac{1}{\rho}}\right)^{\frac{\rho}{1-\rho}}}-\kappa\geq0
\end{align}

The first-order conditions are the same as under monopoly (\ref{eq:L0},
\ref{eq:L1}, \ref{eq:L2}). Combining these with the zero-profit
constraint, we get the following solution. If $\rho<1$: 
\begin{equation}
C_{0}^{N}=\left(\frac{y-\kappa}{2+\beta^{\frac{1}{\rho}}},0,\left(\frac{1+\beta^{\frac{1}{\rho}}}{\beta}\right)^{\frac{1}{1-\rho}}\left(\frac{y-\kappa}{2+\beta^{\frac{1}{\rho}}}\right)\right)\label{eq:naive-comp-contract1}
\end{equation}

If $\rho>1$, the solution is undefined, but in the limit is given
by: 
\begin{equation}
C_{0}^{N}=\left(\frac{y-\kappa}{1+\beta^{\frac{1}{\rho}}\left(1+\beta^{\frac{1}{\rho}}\right)},\beta^{\frac{1}{\rho}}\left(1+\beta^{\frac{1}{\rho}}\right)^{\frac{1}{1-\rho}}\left(\frac{y-\kappa}{1+\beta^{\frac{1}{\rho}}\left(1+\beta^{\frac{1}{\rho}}\right)}\right),\infty\right)\label{eq:naive-comp-contract2}
\end{equation}

\emph{Proof of Proposition 5:}
\begin{enumerate} [label=\alph*)]

\item Under non-exclusive contracts, firms offering period 0 contracts
do not benefit from renegotiation (profits from renegotiation will
equal $\kappa$). So the equilibrium contract is the one that is arrived
at without taking renegotiation into account\textendash i.e. the full-commitment
contract. If $\kappa<\bar{\kappa}$, the gains from renegotiation
exceed the transaction costs, so the contract will be renegotiated.

\item The following observations establish part (b). First, $U_{0}\left(C_{0}^{N}\right)$
is strictly falling in $\kappa$ while $U_{0}\left(C_{0}^{F}\right)$
is invariant in $\kappa$. Second, at $\kappa=\bar{\kappa}$, $U_{0}\left(C_{0}^{N}\right)>U_{0}\left(C_{0}^{F}\right)$
(this must be true by construction of $C_{0}^{N}$). Third, if $\kappa$
gets indefinitely large, $U_{0}\left(C_{0}^{N}\right)<U_{0}\left(C_{0}^{F}\right)$,
so Zero-self will prefer the full-smoothing commitment contract over
the renegotiable contract.

Suppose $\rho<1$. Comparing $C_{0}^{F}$ (\ref{eq:c-f}) to $C_{0}^{N}$
(\ref{eq:naive-comp-contract1}), it is clear that $c_{0}^{N}<c_{0}^{F}$.
Suppose $\rho>1$. If $\kappa$ is small enough, $c_{0}^{N}>c_{0}^{F}$.
$\Square$
\end{enumerate}

\subsection{Nonprofits}

\emph{Proof of Proposition 6:}
\begin{enumerate} [label=\alph*)]
\item A non-profit will earn higher raw
profits $\Pi$ than a for-profit. If $f\left(\Pi\left(C_{0}^{mNP};Y_{0}\right)\right)\geq\Pi\left(C_{0}^{mP};Y_{0}\right)$
(i.e. if the captured profit function has a slope sufficiently close
to 1 up to $\Pi\left(C_{0}^{mNP};Y_{0}\right)$), the firm will choose
to operate as a nonprofit.

\item
\begin{enumerate} [label=\roman*)]
\item Suppose all firms are for-profit and offer the renegotiation-proof
contract $C_{0}^{P}$. There is some $\varepsilon_{1}$ and $\varepsilon_{2}$
satisfying $0<\varepsilon_{2}<\varepsilon_{1}$ and a corresponding
$\hat{C}_{0}=\left(c_{0}^{P},c_{1}^{P}-\varepsilon_{1},c_{2}^{P}+\varepsilon_{2}\right)$
such that $U_{0}\left(\hat{C}_{0}\right)=U_{0}\left(C_{0}^{P}\right)$
and 
\[
f\left(\Pi_{0}\left(\left(\hat{C}_{0}\right);Y_{0}\right)+\Pi_{1}\left(C_{1}^{m1}\left(\hat{C}_{1}\right);\hat{C}_{1}\right)\right)<\kappa
\]

So, any firm can make positive profits by operating as a non-profit.
Therefore, in equilibrium, consumers will borrow only from non-profit
firms.

\item If all firms are nonprofit, an individual firm has a strict incentive
to switch to for-profit status, and make profits in period 1. Therefore,
there must be for-profits in equilibrium, and equilibrium contracts
will be constrained by their presence. $\Square$
\end{enumerate}
\end{enumerate}

%\bibliographystyle{aer}
\bibliographystyle{authordate1}
\bibliography{renegotiation}

\end{document}

